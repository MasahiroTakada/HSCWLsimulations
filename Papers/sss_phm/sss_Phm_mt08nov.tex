%\documentclass[prd,twocolumn,amsmath,amssymb,floatfix,superscriptaddress]{revtex4-1}
%\documentclass[prd,onecolumn,amsmath,amssymb,floatfix,superscriptaddress]{revtex4-1}

%\usepackage{graphicx}
%\usepackage{amssymb}
%\usepackage{amsmath}
%\usepackage{bm}
%\usepackage{hyperref}
%\DeclareMathOperator{\tr}{tr}
%\DeclareMathOperator{\sgn}{sgn}
%\usepackage{color}

\documentclass[onecolumn,notitlepage,showpacs,amsmath,amssymb,prd,floatfix]{revtex4-1}
\usepackage{graphicx}
\usepackage{bm}
\usepackage{amsmath}
\usepackage[varg]{txfonts}
\usepackage{natbib}
\usepackage{color}


\DeclareMathAlphabet\mathbfcal{OMS}{cmsy}{b}{n}
\definecolor{darkgreen}{cmyk}{0.85,0.2,1.00,0.2} 
\definecolor{purple}{cmyk}{0.5,1.0,0,0} 
\def\physrep{Phys.~Rep.}

% default highlighting for our editing purposes
\newcommand{\mtrv}[1]{{\textcolor{blue}{#1}}}
\newcommand{\hidden}[1]{\textcolor{purple}{#1}}
\newcommand*\widebar[1]{\@ifnextchar^{{\wide@bar{#1}{0}}}{\wide@bar{#1}{1}}}
%

% hidden highlighting for distribution
%\newcommand{\wh}[1]{{#1}}
%\newcommand{\mtrv}[1]{{#1}}
%\newcommand{\yl}[1]{{#1}}
%\newcommand{\hidden}[1]{}
%

\newcommand{\parg}{}



\newcommand{\tX}{{X}}

\def\barray{\begin{array}} 
\def\earray{\end{array}}
\def\be{\begin{equation}}
\def\ee{\end{equation}}
\def\ben{\begin{equation} \nonumber}
\def\een{\end{equation}}
\def\ban{\begin{eqnarray*}}
\def\ean{\end{eqnarray*}}
\def\ba{\begin{eqnarray}}
\def\ea{\end{eqnarray}}

\def\ave#1{\left\langle #1 \right\rangle}


\def\({\left(}
\def\){\right)}
\def\half{{1\over2}}

\newcommand{\simgt}{\lower.5ex\hbox{$\; \buildrel > \over \sim \;$}}
\newcommand{\simlt}{\lower.5ex\hbox{$\; \buildrel < \over \sim \;$}}


\newcommand{\mnras}{Mon.~Not.~Roy.~Astron.~Soc.}
%\newcommand{\physrep}{Phys.~Rep.}
\newcommand{\jcap}{JCAP}
\newcommand{\pasj}{PASJ}
\newcommand{\apjl}{\apj~Lett.}
\newcommand{\aap}{Astronomy \& Astrophysics}

\newcommand{\tr}[1]{[#1]}
\newcommand{\gisig}{ {\boldsymbol{\Sigma}}}
\newcommand{\sgisig}{ {\boldsymbol{\gamma}}}
\newcommand{\ul}[3]{#1^{#2}_{\hphantom{#2}#3}}
\newcommand{\lu}[3]{#1_{#2}^{\hphantom{#2}#3}}
\newcommand{\fid}{\Sigma}
\newcommand{\bgamma}{\boldsymbol{\gamma}}
\newcommand{\bfid}{\boldsymbol{\Sigma}}
\newcommand{\bg}{{\bf g}}
\newcommand{\bmin}{\boldsymbol{\eta}}   
\newcommand{\blambda}{\boldsymbol{\Lambda}}   
\newcommand{\bvier}{{\bf e}}
\newcommand{\bfvier}{{\bf L}}

\newcommand{\bx}{{\bf x}}
\newcommand{\by}{{\bf y}}
\newcommand{\bk}{{\bf k}}
\newcommand{\btheta}{{\bm{\theta}}}
\newcommand{\bq}{{\bf q}}
\newcommand{\bl}{{\bf l}}
\newcommand{\bt}{{\bf \theta}}
\newcommand{\lin}{{\rm L}}
\newcommand{\non}{{\rm N}}
\newcommand{\tdelta}{\tilde{\delta}}
\newcommand{\tW}{\tilde{W}}
\newcommand{\tu}{\tilde{u}}
\newcommand{\fpole}{{\cal F}}
\newcommand{\dr}{\mathrm{d}}

\newcommand{\deltab}{\delta_{\rm b}}
\newcommand{\deltam}{\delta_{\mathrm{m}}}
\newcommand{\deltaml}{\delta_{\mathrm{m,lin}}}
\newcommand{\tdeltam}{\tilde{\delta}_{\mathrm{m}}}
\newcommand{\tdeltaml}{\tilde{\delta}_{\mathrm{m,lin}}}
\newcommand{\deltah}{\delta_{\mathrm{h}}}
%\newcommand{\hphm}{\hat{P}_{\mathrm{hm}}}
\newcommand{\phm}{P_{\mathrm{hm}}}
\newcommand{\phml}{P_{\mathrm{hm}}^{\rm lin}}
\newcommand{\hphm}{\hat{P}_{\mathrm{hm}}}
\newcommand{\chm}{C_{\mathrm{hm}}}
\newcommand{\hchm}{\hat{C}_{\mathrm{hm}}}
\newcommand{\phh}{P_{\mathrm{hh}}}
\newcommand{\chh}{C_{\mathrm{hh}}}
\newcommand{\hphh}{\hat{P}_{\mathrm{hh}}}
\newcommand{\hchh}{\hat{C}_{\mathrm{hh}}}
\newcommand{\bh}{\mathrm{h}}
\newcommand{\bthmhm}{\bar{T}_{\rm hmhm}}
\newcommand{\bnh}{\frac{\dr n}{\dr M}}
\newcommand{\bnhs}{\bar{n}_{\rm h}^S}
\newcommand{\bnhd}{\frac{\dr n}{\dr M'}}
%\bar{n}_{\mathrm{h}}}
\newcommand{\pml}{P^{\rm lin}_{\rm m}}
\newcommand{\sigmacri}{\Sigma_{\rm cr}^{-1}}
\newcommand{\sigmacr}{\Sigma_{\rm cr}}
\newcommand{\sigmah}{\Sigma_{\rm h}}
\newcommand{\hNh}{\hat{N}_{\rm h}}
\newcommand{\hdNh}{\widehat{\delta N}_{\rm h}}
\newcommand{\bNh}{\bar{N}_{\rm h}}
\newcommand{\tdsigmah}{\widetilde{\delta N}_{\rm h}\!}
\newcommand{\tsigmah}{\tilde{\Sigma}_{\rm h}}
\newcommand{\tdeltah}{\tilde{\delta}_{\rm h}}
\newcommand{\dsigma}{\Delta\Sigma}
\newcommand{\hdsigma}{\widehat{\dsigma}}
\newcommand{\havedsigma}{\widehat{\ave{\dsigma}}}
\newcommand{\hnh}{\hat{n}_{\mathrm{h}}}
\newcommand{\hdnh}{\widehat{\delta n}_{\mathrm{h}}}
\newcommand{\dnh}{\delta n_{\mathrm{h}}}
\newcommand{\tdnh}{\widetilde{\delta n}_{\mathrm{h}}}

\newcommand{\Omegam}{{\Omega_\textrm{m}}}
\newcommand{\Omegab}{{\Omega_\textrm{b}}}
\newcommand{\Omegac}{{\Omega_\textrm{c}}}
\newcommand{\OmegaL}{{\Omega_\Lambda}}
\newcommand{\OmegaK}{{\Omega_K}}
\newcommand{\As}{{A_\textrm{s}}}
\newcommand{\lnAs}{{\ln\!A_\textrm{s}}}
\newcommand{\ns}{{n_\textrm{s}}}
\newcommand{\br}{{\rm b}}
\newcommand{\hMpci}{$h\,$Mpc$^{-1}$}
\newcommand{\hiMpc}{$h^{-1}\,$Mpc}

\newcommand{\Dv}{{\cal P}}

\def\avrg#1{\left\langle #1 \right\rangle}
\def\avrgb#1{\left\langle #1 \right\rangle_\textrm{b}}


\begin{document}

\title{Super sample variance of stacked lensing}

\author{Masahiro Takada$^1$, Takahiro Nishimichi$^{1,2}$, Ryuichi
Takahashi$^{3}$, Masamune Oguri$^2$ TBD}
%
\affiliation{$^1$Kavli Institute for the Physics
and Mathematics of the Universe (Kavli IPMU, WPI), The University of
Tokyo, Chiba 277-8583, Japan}
\affiliation{$^2$Physics Department, The University of Tokyo, Bunkyo, Tokyo 113-0031, Japan}
\affiliation{$^3$Department of Physical Science, Hiroshima University, 1-3-1 Kagamiyama,
Higashi-Hiroshima, Hiroshima 739-8526, Japan}

 \begin{abstract}
  Title is still tempolary... a better title we should come up with
  later. 
To be filled ... 
 \end{abstract}
\maketitle

\section{Introduction}

\cite{HuKravtsov:03}
\cite{TakadaBridle:07}
\cite{TakadaJain:09}
\cite{TakadaHu:13} \cite{Lietal:14a} \cite{Lietal:14b}
\cite{OguriTakada:11}
\cite{Hikageetal:13} \cite{Hikageetal:12}
\cite{TakadaSpergel:13} \cite{Schaanetal:14} \cite{Miyatakeetal:15a}
\cite{Miyatakeetal:15}
\cite{Satoetal:09} \cite{Okabeetal:10}
\cite{Lietal:15} \cite{Baldaufetal:15}


\section{Covariance of stacked lensing in a survey window}

%\subsection{Survey window}
%
%Assume we measure observables, the halo density field and the lensing
%field, through a survey window $W(\btheta)$, which is 1 in the measured
%region and 0 in the unmeasured region. The survey area is
%%
%\begin{equation}
% \Omega_S\equiv \int\!\dr^2\btheta~W(\btheta).
%\end{equation}
%%
%In the following, we express our observables taking into account the
%effect of survey window. 


\subsection{Halo statistics: number counts and projected power spectrum}

First we consider statistical observables of halos. We assume that halos
in a survey region can be identified from observables such as optical
richness and X-ray observables, and then assume that halo mass and
redshift of each halo are available. However, the following discussion
can be extended to a more generally case where only their proxies
are available.
%can be obtained for each halo.

%Following the formulatoin in Ref.~\cite{HuKravtsov:03}, we assume that
%that
The number density fluctuation field for halos in the mass range
$[M,M+\dr M]$ is given as
%to the underlying density field via a linear halo bias as
%
\begin{equation}
 n_\bh(\bx; M)\dr M\simeq
  %\bar{n}_\bh(M)\dr M
  \frac{\dr n}{\dr M}
  \left[1+\deltah(\bx; M)\right]
\end{equation}
%
where
%$\bnh$
$dn/dM$ is the (ensemble-average) number density of halos with masses
 $[M,M+\dr M]$, and $\deltah(\bx)$ is the 3D number density fluctuation
 field of the halos.


Denoting the radial selection function by $f_\bh(\chi;z_L)$ and the halo
mass selection at each redshift by $S(M;\chi)$, we can define the
projected number density field of halos, integrated over ranges of
redshift and halo masses, in terms of the 3D density field $n_{\rm
h}(\bx;\chi)$ as
%
\begin{eqnarray}
 N_{\rm h}(\bx_\perp)&=&\int\!\dr\chi~f_\bh(\chi;z_L)\int\!\dr M~S(M;\chi)W(\bx_\perp)
  n_\bh(\bx;M,\chi)\nonumber\\
 &=&\int\!\dr\chi~f_\bh(\chi;z_L)\int\!\dr M~S(M;\chi)\frac{\dr n}{\dr M}
  \left[1+W(\bx_\perp)\deltah(\chi,\bx_\perp)\right],
  \label{eq:Nh}
\end{eqnarray}
%
where we introduced the notation ``$z_L$'' in $f_\bh(\chi)$ to
explicitly mean that the radial selection is non-zero around a target
redshift $z_L$, which will be below taken for the redshift slice of
lensing halos.  $N_{\rm h}(\bx_\perp)$ is the projected number density
field in dimension of [Mpc$^{-2}$], not an angular number density, and
we throughout this paper employ distant observer approximation as well
as flat-sky approximation.  The selection functions are generally
defined so as to satisfy $0\le f_{\rm h}\le 1$ and $0\le S(M;\chi)\le
1$. For an ideal, homogeneous complete selection, $f_{\rm h}=0$ or $1$
and similar for $S(M)$.  The vector $\bx_\perp$ in the above equation
denotes the position vector in the two-dimensional plane at distance
$\chi$, perpendicular to the line-of-sight direction, which is converted
from the angular position $\btheta$ via relation $\bx_\perp\equiv
\chi\btheta$. $W(\btheta)$ is a survey window, which is 1 in the
measured region and 0 in the unmeasured region. Hence the survey area is
given as
%
\begin{equation}
 \Omega_S\equiv \int\!\dr^2\btheta~W(\btheta).
\end{equation}
%


The mean surface density in the finite-area survey region is defined by
the survey window average of Eq.~(\ref{eq:Nh}):
%
\begin{eqnarray}
 \hNh(z_L)=\frac{1}{\Omega_S}\int\!\dr^2\btheta~N_{\rm h}(\bx_\perp)
%&=&\int\!\dr\chi~f_\bh(\chi;z_L)\hnh^S(\chi)
%\nonumber\\
 &\simeq &\int\!\dr\chi~f_\bh(\chi;z_L)\int\!\dr M~S(M;\chi)\frac{\dr
 n}{\dr M}\left[
	       1+\frac{b(M)}{\chi^2\Omega_S(\chi)}\int\!\dr^2\bx_\perp~W(\bx_\perp)
	       \deltaml(\chi,\bx_\perp)\right]\nonumber\\
&=&\int\!\dr\chi~f_\bh(\chi;z_L)\int\!\dr M~S(M;\chi)\frac{\dr n}{\dr
 M}\left[1+b(M)\deltab(\chi)\right]\nonumber\\
&=& \bNh\left[1+\frac{1}{\bNh}\int\!\!\dr\chi~f_{\rm h}(\chi)\bnhs\bar{b}^S\deltab\right].
\label{eq:hNh}
\end{eqnarray}
%
The equality $\deltab$ is the background density contrast
in each redshift slice at distance $\chi$:
%
\begin{equation}
\deltab(\chi)\equiv
 \frac{1}{\Omega_S}\int\!\dr^2\btheta~W(\btheta)\deltaml(\chi,\chi\btheta)
 =\frac{1}{\chi^2\Omega_S(\chi)}\int\!\dr^2\bx_\perp ~W(\bx_\perp)\deltaml(\chi,\bx_\perp),
\label{eq:deltab}
\end{equation}
%
where we have again used the conversion $\bx_\perp =\chi\btheta$, and
$\chi^2\Omega_S$ is the effective area in the perpendicular plane at
distance $\chi$, in units of [${\rm Mpc}^2$]. We here considered the
background mode at each redshift that is defined by the survey window
average of Fourier modes in the two-dimensional plane perpendicular to
the line-of-sight direction. Since we will later employ the Limber's
approximation, we throughout this paper ignore the radial mode of the
background mode which is a good approximation for the lensing
statistics. In the second equality of Eq.~(\ref{eq:hNh}), we defined the
mean 3D number density of halos, averaged by the survey window, at
distance $\chi$:
%
\begin{equation}
 \hnh^S(\chi)\equiv \frac{1}{\chi^2\Omega_S}\int\!\!\dr^2\bx_\perp~
  W(\bx_\perp)\int\!\!\dr M~
  S(M)n_{\bh}(\bx; M, \chi)=
  \int\!\!\dr M~\bnh S(M)b(M)\deltab
  \label{eq:hnhS}
\end{equation}
%
Quantities with hat $\hat{\hspace{1em}}$ notation, here and hereafter,
denote their estimators.  Here we assumed that the halo density field at
large scales, after the window average, is related to the underlying
mass density field as $\deltah(\bx; M)\simeq b(M)\deltaml(\bx)$, where
$\deltaml(\bx)$ is the mass density fluctuation field in the linear
regime and $b$ is the linear halo bias, assuming that the angular
average over a sufficiently wide area smooths out the small-scale
fluctuations. On the r.h.s. of Eq.~(\ref{eq:hNh}) we have defined the
ensemble-average halo number density and bias, integrated over the halo
mass range, in each redshift slice at $z=z(\chi)$:
%
\begin{eqnarray}
 \bnhs(\chi) &\equiv & \int\!\!\dr M~ \frac{dn}{dM}S(M), \nonumber\\
 \bar{b}^S(\chi)&\equiv & \frac{1}{\bnhs}\int\!\!\dr M~
  \frac{dn}{dM}S(M)\bnh b.
\label{eq:aveb}
\end{eqnarray}
%
Furthermore, $\bNh$ in Eq.~(\ref{eq:hNh}) is the ensemble average or
global mean of the surface halo number density:
%
\begin{equation}
 \bNh(z_L)\equiv 
 \ave{\hNh(z_L)}=\int\!\dr\chi~f_\bh(\chi;z_L)\int\!\dr
 M~S(M;\chi)\frac{\dr n}{\dr M}, 
\end{equation}
%
where we have used $\ave{\deltab}=0$.  Here and hereafter quantities
with bar notation denote their ensemble average quantities.


%$b(M)$ is the linear bias for halos of mass $M$, $\deltaml(\bx)$
%is the three-dimensional mass fluctuation field, and $\bar{n}_{\bh}$ is
%the {\em global} mean number density, i.e. the halo mass function. Note
%that, assuming that the above equation is valide for a coase-grained
%viewpoint and therefore $\deltaml$ is from long-wavelength modes and in
%the linear regime, we
%consider up to the first-order bias or equivalently ignored the
%higher-order bias contributions.


The projected number density fluctuation field of halos is
%
\begin{equation}
 \hdNh(\bx_\perp)\equiv
    \int\!\dr\chi~f_\bh(\chi;z_L)\int\!\dr M~S(M)\bnh
  W(\bx_\perp)\deltah(\chi,\bx_\perp; M).
\end{equation}
%
The Fourier transform is
%
\begin{equation}
 \tdsigmah(\bk_\perp)\equiv \int\!\dr^2\bx_\perp~ \hdNh(\bx_\perp)e^{-i\bk_\perp\cdot\bx_\perp}
  =\int\!\dr\chi~f_\bh(\chi;z_L)\int\!\dr M~\bnh S(M)\int\!\frac{\dr k_\parallel}{2\pi}
  \tdeltah^W\!(k_\parallel,\bk_\perp)e^{ik_\parallel\chi}
\end{equation}
%
where 
%
\begin{equation}
  \tdeltah^W\!(k_\parallel,\bk_\perp)\equiv 
  \int\!\frac{\dr^2\bq}{(2\pi)^2}\tW(\bq)
  \tdeltah(k_\parallel,\bk_\perp-\bq),
\end{equation}
%
$\tW(\bq;\chi)\equiv
\int\!\dr^2\bx_\perp~W(\bx)e^{-i\bq\cdot\bx_\perp}$, and
quantities with tilde symbol hereafter denote their Fourier
transforms.

Hence we can define as estimator of the projected power spectrum of
halos as
%
\begin{eqnarray}
\hchh(k_{\perp,i})\equiv
 \frac{1}{\hNh(z_L)^2}\int\!\dr\chi\int\!\dr\chi'
 f_\bh(\chi)f_\bh(\chi')
 \int\!\dr M\int\!\dr M'~S(M)\bnh S(M')\bnhd\nonumber\\
 \times
\frac{1}{\chi^2\Omega_S}\int_{|\bk_\perp|\in k_{\perp i}}\!\frac{\dr^2\bk_{\perp}}{A(k_{\perp
i})}\int\!\frac{\dr k_\parallel}{2\pi}\int\!\frac{\dr k'_{\parallel}}{2\pi}
~\tdeltah^W\!(k_\parallel,\bk_\perp)\tdeltah^W\!(k'_\parallel,-\bk_\perp)
e^{ik_\parallel\chi+ik'_\parallel\chi'},
\label{eq:e_chh}
\end{eqnarray}
%
where the integral $\int\!\!\dr^2\bk_\perp$ is over an annulus in
$\bk_\perp$ space of width $\Delta k$, and the area $A(k_{\perp
i})\simeq 2\pi k_{\perp i}\Delta k$.  Using the halo power spectrum
definition,
%
\begin{equation}
 \ave{\deltah(k_\parallel,\bk_\perp;M)\deltah(k'_\parallel,\bk'_{\perp};M')}=(2\pi)^3
  \delta_D(k_\parallel+k'_\parallel)\delta_D^2(\bk_\perp+\bk'_\perp)\phh(k;M,M'), 
\end{equation}
%
and using the Limber's approximation \cite{Limber:54},
we can compute the ensemble
average of the above power spectrum estimator:
%
\begin{eqnarray}
\chh(k_i)=\ave{\hchh}(k_{\perp, i})&=&
   \frac{1}{\hNh(z_L)^2}\int\!\dr\chi\int\!\dr\chi'
 f_\bh(\chi)f_\bh(\chi')
 \int\!\dr M\int\!\dr M'~S(M)\bnh S(M')\bnhd\nonumber\\
&&\hspace{4em} \times
\frac{1}{\chi^2\Omega_S}\int\!\frac{\dr^2\bk_{\perp}}{A(k_{\perp
i})}\int\!\frac{\dr^2\bq}{(2\pi)^2}
%~\tdeltam^W(k_\parallel,\bk_\perp)\tdeltam^W(-k_\parallel,-\bk_\perp)
\phh\left(\sqrt{k_\parallel^2+|\bk_\perp-\bq|^2}\right)
|\tW(\bq)|^2e^{ik_\parallel(\chi-\chi')}
\nonumber\\
 &\simeq&\frac{1}{\bNh(z_L)^2}\int\!\dr\chi
  f_\bh(\chi)^2\int\!\dr M\int\!\dr M'~S(M)\bnh S(M')\bnhd
  \phh(k_{\perp,i},\chi;M,M')
  \frac{1}{\chi^2\Omega_S}\int\!\frac{\dr^2\bq}{(2\pi)^2}|\tW(\bq)|^2
\nonumber\\
 &=&\frac{1}{\bNh(z_L)^2}\int\!\dr\chi
  f_\bh(\chi)^2\int\!\dr M\int\!\dr M'~S(M)\bnh S(M')\bnhd
  \phh(k_{\perp,i},\chi; M, M').
  \label{eq:ave_hchh}
\end{eqnarray}
%
Here we have used that $\phh(|\bk_\perp-\bq|)\simeq \phh(k_\perp)$ over
the integration range of $\dr^2\bq$ which the window function supports
and also assumed that $\phh(k)$ is not a rapidly varying function within
the $k_\perp$-bin ($k_\perp \gg q$). Here and hereafter we will often
omit ``$_{\perp}$'' in $k_{\perp}$ for notational simplicity. In the
third equality on the r.h.s., we used the general identity for the
window function:
%
\begin{equation}
 \chi^2\Omega_S=\int\!\dr^2\bx_\perp~ W(\bx_\perp)^n=
  \int\!\!\left[\prod_{a=1}^n\frac{d\bq_a}{(2\pi)^2}\tW(\bq_a)\right](2\pi)^2
  \delta_D^2(\bq_{1\dots n}),
\end{equation}
%
where $\bq_{1\dots n}\equiv \bq_1+\bq_2+\cdots+\bq_n$. For $n=2$,
$\chi^2\Omega_S=\Omega_S\int\!\dr^2\bq/(2\pi)^2 ~ |\tW(\bq)|^2$.
%the identity
%$\int\!\dr^2\bq/(2\pi)^2~|\tW(\bq)|^2=\chi^2 \Omega_s$, derived from the fact
%$\int\!\dr^2\bx_\perp~ W(\btheta)^n=\chi^2\Omega_S$ with
%$\bx_\perp=\chi\btheta$.
%
%\begin{equation}
%\Omega_S=\int\!d\btheta~W(\btheta)=\int\!d\btheta~W(\btheta)^n
 %\end{equation}
If we employ the halo model, where the halo power spectrum is modeled as
$\phh(k; M, M')\simeq b(M)b(M')\pml(k)$, the projected power spectrum is
rewritten as
%
\begin{equation}
 \chh(k)=\frac{1}{\bNh(z_L)^2}\int\!\dr\chi
  f_\bh(\chi)^2 \left[\bnhs\bar{b}^S\right]^2
  \pml(k;\chi),
 % \left[\int\!\!\dr M~\bnh S(M)b(M)\right]^2
 % \pml(k;\chi)
\end{equation}
%
where $\pml(k)$ is the linear mass power spectrum. The projected power
spectrum $\chh$ has a dimension of $[{\rm Mpc}^2]$.


\subsection{Stacked lensing: An estimator of the projected halo-mass cross-correlation}

The lensing convergence field at an angular position $\btheta$ on the sky
arises from a projection of the three-dimensional mass fluctuation field
$\deltam$, weighted by the lensing efficiency kernel:
%
\begin{equation}
 \kappa^W\!(\btheta)\equiv W(\btheta)\kappa(\btheta)=
  \int\!\dr\chi~
 f_\kappa(\chi)W(\btheta)\deltam(\chi,\chi\btheta;z)
 =\int\!\dr\chi~
 f_\kappa(\chi)
 \int\!\frac{\dr^2\bk_\perp}{(2\pi)^2}
 \frac{\dr k_\parallel}{2\pi}\tilde{\delta}_{\rm m}^W\!(k_\parallel,\bk)e^{ik_\parallel\chi+i\bk_\perp\cdot\bx_\perp}
\label{eq:wkappa}
\end{equation}
%
where $\bx_\perp=\chi\btheta$, $\delta_m(\bx)$ is the three-dimensional
mass fluctuation field, and the lensing radial kernel $f_\kappa(\chi)$
is given as
%
\begin{equation}
 f_\kappa(\chi)=\bar{\rho}_{\rm m0} \sigmacri(\chi).
  \label{eq:fkappa}
\end{equation}
%
The quantity $\sigmacr$ is the critical surface mass density defined for
each observer-lens-source system:
%
\begin{equation}
\sigmacri(z)\equiv 4\pi Ga^{-1}\chi(z)\left[1-\chi(z)\ave{\frac{1}{\chi_s}}\right],
\end{equation}
%
with 
$\ave{1/\chi_s}=
\left.\int_{z_{\rm s,min}}^\infty\!dz_s
 p(z_s)1/\chi(z_s)\right/
 \int_{z_{\rm s,min}}^\infty\!dz_s p(z_s)$,
%
where $p(z_s)$ is the redshift distribution of source galaxies, and
$z_{\rm s,min}$ is the minimum redshift which an observer determines to
select source galaxies behind lensing halos at $z_L$.  The factor of
$\bar{\rho}_{\rm m0}$ in Eq.~(\ref{eq:wkappa}), instead of the mean
density $\bar{\rho}_{\rm m}(z)$ at redshift $z$, is from our use of the
comoving coordinates.

Before going to the detailed calculation, it would be illustrative to
reexpress the convergence field as a sum of two contributions:
%
\begin{eqnarray}
\kappa(\btheta)&\simeq&
% \nonumber\\
% &=&
% =
 \bar{\rho}_{\rm
 m0}\sigmacri(z_L)\int_{z_L}\!\!\dr\chi~\deltam(\chi_L,\chi_L\btheta)+
\bar{\rho}_{\rm m0}\int_{z\ne z_L}\!\dr\chi~
\sigmacri(z)\deltam(\chi,\chi \btheta).
\label{eq:kappa_cocept}
\end{eqnarray}
%
The first term on the r.h.s. denotes a lensing contribution from the
mass distribution  at the lens redshift, assuming a thin redshift slice
where the critical density $\sigmacr$ is not largely varying,
and the second term is the lensing
contribution from the mass distribution at different redshifts, which we
call the cosmic shear contribution.

The stacked lensing for sampled halos with known redshifts probes the
average mass distribution surrounding the lensing halos. Observationally
the stacked lensing profile can be estimated by averaging shapes of
background galaxies for all the pairs of halos and source galaxies
\cite{Miyatakeetal:15,Niikuraetal:15}:
%
\begin{equation}
 \widehat{\ave{\dsigma}}(R)\equiv\frac{1}{N_{\rm pairs}}
	\left.\sum_{{\rm all~ pairs}; (l,s)}\sigmacr(z_{l},z_s)\epsilon_{+(s)}
	\right|_{R=\chi(z_{l})\Delta\theta_{ls}},
\end{equation}
%
where the summation runs over all the halo and source galaxy pairs each
of which is separated by a particular projected distance $R$ in the lens
plane of each halo at $z_l$, and $\epsilon_+$ is the tangential
component of source galaxy ellipticity with respect to the halo
center. $\Delta\theta_{ls}$ is the angular separation between lens and
source in each pair, and $\chi(z_l)$ is the distance to halo at redshift
$z_l$.  Here we assume that redshift information of each source galaxy
is also available, e.g., based on the photometric redshift estimation.
Thus, {\em a priori} knowledge of each lens redshift allows us to probe
the average projected mass profile as a function of comoving distances
in each lens plane, rather than angular radii
\cite{Hikageetal:13,Nishizawaetal:13}.  For simplicity we do not
consider a weighting in the above average, which is often employed in an
actual measurement in order to optimize the signal-to-noise ratio
\citep[e.g.,][]{Miyatakeetal:15}.

The ensemble-average expectation, i.e. the cosmological signal of the
stacked lensing arises from the cross-correlation between the projected
number density fluctuation field of lensing halos and the lensing shear
field:
%
\begin{equation}
{\ave{\dsigma}}(R)=\frac{1}{\bNh(z_L)}
  \ave{\delta N_{\rm h}(\bx_\perp)\sigmacr(\chi)
  \gamma_{+}(\bx'_\perp)}_{R=|\bx_\perp-\bx'_\perp|;\bx'_\perp=\chi\btheta},
  \label{eq:e_dsigma}
\end{equation}
%
where $\gamma_+(\bx_\perp^{\prime})$ is the tangential component of the
shear field with respect to halo center, in the angular direction of
$\bx_\perp^{\prime}$ via the relation $\bx_\perp^{\prime}=\chi\btheta$.
The average of non-lensing mode $\gamma_\times$, which is the
45-degrees-rotated shear component relative to $\gamma_+$, is
vanishing,$\ave{\delta N_{\rm h}\sigmacr \gamma_\times}=0$, and
therefore this can be used as a diagnostic of residual systematic errors
such as an error due to an imperfect shape measurement. If defining the
projected cross-power spectrum of halos and the surrounding mass
distribution, the stacked lensing profile is expressed as
%
\begin{equation}
 \ave{\dsigma}(R)=\int\!\!\frac{kdk}{2\pi}~\chm(k)J_2(kR),
\end{equation}
%
where $J_2(x)$ is the 2nd-order Bessel function and the cross-power
spectrum is defined as
%
\begin{eqnarray}
 \chm(k)&=&\frac{\bar{\rho}_{\rm m0}}{\bNh(z_L)}\int\!\!\dr\chi~f_{\rm
  h}(\chi)\int\!\!\dr M~\bnh S(M)P_{\rm hm}(k;M,\chi)
  =\frac{\bar{\rho}_{\rm m0}}{\bNh(z_L)}\int\!\!\dr\chi~f_{\rm
  h}(\chi)\bnhs\phm^S(k;\chi),
  \label{eq:chm}
\end{eqnarray}
%
where $\phm(k;M,\chi)$ is the 3D cross-power spectrum between the mass
distribution and halos of mass $M$ and at redshift $z=z(\chi)$,
and $\phm^S(k;\chi)$ is the cross-power spectrum integrated by the halo
mass function, weighted by the halo mass selection:
%
\begin{equation}
 \phm^S(k;\chi)\equiv \frac{1}{\bnhs(\chi)}\int\!\!\dr M~\bnh S(M)\phm(k;M).
\label{eq:phmS}
\end{equation}
%
If we employ the halo model, we can express $\chm(k)$ by a sum
of the 1- and 2-halo terms:
%
\begin{equation}
 \chm(k)=\chm^{\rm 1h}(k)+\chm^{\rm 2h}(k),
\end{equation}
%
with
%
\begin{eqnarray}
\chm^{\rm 1h}(k)&\equiv &
  \frac{\bar{\rho}_{\rm m0}}{\bNh(z_L)}\int_{\chi_L}\!\dr\chi ~f_\bh(\chi)
  \int\!\dr M~\bnh S(M,z_L)\frac{M}{\bar{\rho}_{\rm m0}}u_M(k;\chi)
  =\frac{\bar{\rho}_{\rm m0}}{\bNh(z_L)}\int_{\chi_L}\!\dr\chi
  ~f_\bh(\chi)\bnhs{\cal I}^{0}_1(k;\chi),\nonumber\\
%I^{(S)0}_1(k;\chi),\nonumber \\
 \chm^{\rm 2h}(k)&=&
 \frac{\bar{\rho}_{\rm m0}}{\bNh(z_L)}\int_{\chi_L}\!\dr\chi ~f_\bh(\chi)
  \int\!\dr M~\bnh S(M)b(M)\pml(k;\chi)
=\frac{\bar{\rho}_{\rm m0}}{\bNh(z_L)}\int_{\chi_L}\!\dr\chi
~f_\bh(\chi) \bnhs \bar{b}^S
\pml(k;\chi),
%\simeq \bar{b}_1\bar{\rho}_{\rm m0}P^{\rm lin}_{\rm m}(k),\nonumber \\
\end{eqnarray}
%
where we introduced the notations:
%
\begin{eqnarray}
 {\cal I}^{\beta}_\mu(k_1,\cdots,k_\mu)&\equiv&\frac{1}{\bnhs}
  \int\!\dr M\bnh S(M)
  \left(\frac{M}{\bar{\rho}_{\rm m0}}\right)^\mu
  b^\beta\prod_{i=1}^\mu u_M(k_i),
%\nonumber\\
% I^\beta_\mu(k_1,\cdots,k_\mu)&\equiv& \int\!\dr M\bnh
%  \left(\frac{M}{\bar{\rho}_{\rm m0}}\right)^\mu
%b^\beta\prod_{i=1}^\mu u_M(k_i),
\end{eqnarray}
%
where $u_M(k)$ is the Fourier transform of mass density profile for
halos of mass $M$. Note ${\cal I}^{(S)1}_0=\bar{b}^S$ (see Eq.~\ref{eq:aveb}).


Similarly to Eq.~(\ref{eq:e_chh}), taking into account the survey
window, we can define an estimator of the projected cross-power spectrum
as
%
\begin{eqnarray}
 \hchm(k_i)&\equiv & \frac{1}{\hNh(z_L)}\int\!\dr\chi\int\!\dr\chi'~f_\bh(\chi)
  \sigmacr(\chi)
  f_\kappa(\chi')\int\!\dr M~\bnh S(M)
%  \nonumber\\
% &&\times
  \frac{1}{\chi^2\Omega_S}
  \int_{|\bk_\perp|\in k_i}\!\frac{\dr^2\bk_\perp}{A(k_i)}\int\!\frac{\dr k_\parallel}{2\pi}
  \frac{\dr k^{\prime }_\parallel}{2\pi}
  \tdeltah^W\!(k_\parallel,\bk_\perp)\tdeltam^W\!(k'_\parallel,-\bk_\perp)
  e^{ik_\parallel\chi+ ik'_\parallel\chi'}.\nonumber\\
\label{eq:e_chm}
\end{eqnarray}
%
Note that the projected number density $\hNh(z_L)$ in the denominator on
the r.h.s. is an estimator of the mean projected halo number density, and
therefore includes a contribution due to the background density
$\deltab$ (see Eq.~\ref{eq:hNh}). Similarly to Eq.~(\ref{eq:ave_hchh}),
employing the Limber's approximation and assuming that the underlying
power spectrum is not a rapidly varying function within the survey
window, we can check that the ensemble average of the power spectrum
estimator gives the expectation:
%
\begin{eqnarray}
\ave{\hchm(k_i)}&=&\frac{1}{\bNh(z_L)}\int\!\!\dr~f_{\rm
 h}(\chi)\sigmacr(\chi)f_\kappa(\chi)\int\!\!\dr M~\bnh S(M)
 \frac{1}{\chi^2\Omega_S}\int_{|\bk_\perp|\in
 k_i}\!\!\frac{\dr^2\bk_\perp}{A(k_i)}
 \int\!\!\frac{\dr^2\bq}{(2\pi)^2}\phm(|\bk_\perp-\bq|)|\tW(\bq)|^2\nonumber\\
 &\simeq& \frac{\bar{\rho}_{\rm m0}}{\bNh(z_L)}
  \int\!\!\dr\chi~f_{\rm h}(\chi)\int\!\!\dr M \bnh
  S(M)\phm(k_i;M,\chi)=\chm(k_i), 
\end{eqnarray}
%
where we have used $\phm(|\bk_\perp-\bq|)\simeq \phm(k_\perp)$ over the
integration range $\dr^2\bq$ where the survey window supports, and also
assumed that $\phm(k)$ is not a rapidly varying function within the bin
width $\Delta k$ around the $k_i$-bin ($k_\perp \gg q$).

\subsection{Covariance matrix of the stacked lensing power spectrum}

Now we consider the covariance matrix of the projected power spectrum of
stacked lensing, which can be defined in terms of the power spectrum
estimator (Eq.~\ref{eq:e_chm}) as
%
\begin{equation}
 C_{ij}\equiv
  \ave{\hchm(k_i)\hchm(k_j)}-\ave{\hchm(k_i)}\ave{\hchm(k_j)}.
  \label{eq:def_cov}
\end{equation}
%
By inserting the power spectrum estimator (Eq.~\ref{eq:e_chm}) and then
computing the ensemble averages in the above equation, we will below
derive an expression of the covariance matrix. 

First of all, we have to bear in mind that
%the overall normalization of
the stacked lensing estimator depends on the local mean of the projected
halo number density. Eqns.~(\ref{eq:hNh}) and (\ref{eq:hnhS}), we can
find a modulation in the normalization: 
%
\begin{equation}
 \hchm\propto
  \frac{1}{\hNh}\simeq \frac{1}{\bNh}\left(1-\Delta^{\rm h}_{\br}\right)
  = \frac{1}{\bNh}\left[1-\frac{1}{\bNh}
		   \int\!\!\dr\chi~f_{\rm h}(\chi)\frac{\partial \hnh^S}{\partial \deltab}\deltab\right]
\label{eq:def_Deltahb}
\end{equation}
%
where
%
\begin{eqnarray}
 \frac{\partial \hnh^S(\chi)}{\partial \deltab}&=&\frac{\partial }{\partial
  \deltab} \int\!\!\dr M~ \bnh S(M) b(M;\chi)\deltab(\chi) = 
  \int\!\!\dr M~ \bnh S(M) b(M;\chi)= \bnhs(\chi) \bar{b}^S(\chi),\nonumber\\
 \Delta_{\br}^{\rm h}&\equiv& \frac{1}{\bNh}\int\!\!\dr\chi~f_{\rm
  h}(\chi)\int\!\!\dr M~ \bnh S(M) b)
  \deltab=\frac{1}{\bNh}\int\!\!\dr\chi~f_{\rm h}(\chi)\bnhs\bar{b}^S\deltab.
\label{eq:dhnhS_ddb}
\end{eqnarray}
%
%$\Delta_{\br}^{\rm h}\equiv
%(1/\bNh)\int\!\!\dr\chi~\bnhs\bar{b}^S\deltab$.
The background density mode $\deltab$ is a statistical variable, varies
with a survey window, and therefore contributes the covariance.

If we conceptually express the power spectrum estimator as
$\hchm(k_i)\propto (1-\Delta^{\rm h}_{\br})(\hat{h}\hat{\kappa})_{k_i}$,
where we denote the estimators of the halo field and the lensing field
as $\hat{h}$ and $\hat{\kappa}$, the covariance can be expressed by a
sum of three contributions as
%
\begin{eqnarray}
 {\bf C}&\rightarrow& \ave{(1-\Delta^{\rm h}_{\br})^2(\hat{h}\hat{\kappa})_{k_i}(\hat{h}\hat{\kappa})_{k_j}
  }-\ave{(\hat{h}\hat{\kappa})_{k_i}}\ave{(\hat{h}\hat{\kappa})_{k_j}}
%  \nonumber\\
% &&
  =\ave{\hat{h}_{k_i}\hat{\kappa}_{k_j}}\ave{\hat{\kappa}_{k_i}\hat{h}_{k_j}}
  +\ave{\hat{h}_{k_i}\hat{h}_{k_j}}\ave{\hat{\kappa}_{k_i}\hat{\kappa}_{k_j}}
  +\ave{(1-\Delta_{\br}^{\rm h})^2
  (\hat{h}\hat{\kappa})_{k_i}(\hat{h}\hat{\kappa})_{k_j}}_c.
  \label{eq:cov_formal}
\end{eqnarray}
%
The first two terms are from products of the power spectra,
$\ave{\hat{h}\hat{\kappa}}$, $\ave{\hat{h}\hat{h}}$ and
$\ave{\hat{\kappa}\hat{\kappa}}$, which we call the Gaussian covariance
term. The third term arises from the 4-point correlation function or the
trispectrum in the Fourier space, which we call the non-Gasussian term.
If the halo and lensing fields are both Gaussian, e.g. at linear scales,
the non-Gaussian term is vanishing. We generalized the non-Gaussian term
in that it includes the contribution from the modulation in the
projected halo number density. 

Hence, following the method in Ref.~\cite{TakadaHu:13}, the covariance
matrix can be generally expressed as
%by a sum of the three terms:
%
\begin{equation}
 {\bf C}={\bf C}^{\rm Gauss}+{\bf C}^{T_0}+{\bf C}^{\rm SSC}, 
\end{equation}
%
where we have considered the modes satisfying $k_i,k_j\gg
1/(\chi\Omega_{S}^{1/2})$ ($\chi$ is in the range of lens
redshifts). The first term is the Gaussian term as we stated above, the
second term is the non-Gaussian term arising from the trispectrum of
sub-survey modes, and the third term is the super-sample covariance
arising from super-survey modes expressed in terms of $\deltab$ in our
formulation.

\subsubsection{Gaussian covariance}

Extending the formulation in Ref.~\cite{OguriTakada:11} to include the
redshift weight $\sigmacr(\chi)$ in the estimator (Eq.~\ref{eq:e_chm}),
we can derive the Gaussian covariance term corresponding to the first
and second terms in the formal expression (Eq.~\ref{eq:cov_formal}):
%
\begin{equation}
 C^{\rm Gauss}(k_i,k_j)=\frac{1}{N_{\rm
  mode}(k_i)}\delta^K_{ij}\left[\chm(k_i)+C^{\sigmacr}_{\rm
			   hh}(k_i)C^{\rm obs}_\kappa(k_i)\right],
\label{eq:cov_g}
\end{equation}
%
where $\delta^K_{ij}$ is the Kronecker delta function; $\delta^K_{ij}=1$
if $k_i=k_j$ to withing the bin width, otherwise $\delta^K_{ij}=0$. The
number of independent $k$-modes resolved in the shell, by a resolution
of Fourier decomposition of a finite-area survey, is given as
%
\begin{equation}
N_{\rm mode}(k_i)\simeq \frac{A(k_i)S_{\rm
 eff}}{(2\pi)^2}=\frac{A(k_i)\ave{\chi_L^2}\Omega_S}{(2\pi)^2}, 
\end{equation}
%
where $S_{\rm eff}$ is the effective survey area in dimension of $[{\rm
Mpc}^2]$, and $\ave{\chi_L^2}$ is the radial distance to lensing halos,
averaged over the radial selection function: $\ave{\chi_L^2}\equiv
\int\!\!\dr\chi~f_{\rm h}(\chi;z_L)\chi^2/\int\!\!\dr\chi~f_{\rm
h}(\chi;z_L)$. The spectrum $\chh^{\sigmacr}(k)$ is the projected
auto-spectrum of halos including the lensing kernel weight
$\sigmacr(\chi)$ at each lens redshift (see Eq.~\ref{eq:e_dsigma} or
\ref{eq:e_chm}) as well as the shot noise contamination arising from a
finite number of lensing halos used in the stacked lensing measurement,
while $C_\kappa^{\rm obs}(k)$ is the cosmic shear power spectrum
including the intrinsic shape noise contamination arising from a finite
number of source galaxies. These spectra are given as
%
\begin{eqnarray}
\chh^{\sigmacr}&=&\frac{1}{\bNh(z_L)^2}\int\!\!\dr\chi~f_{\rm
 h}(\chi)^2\sigmacr(\chi)^2
 [\bnhs\bar{b}^S]^2\pml(k;\chi) +
 \frac{1}{\bNh(z_L)^2}\int\!\!\dr\chi~f_{\rm
 h}(\chi)\sigmacr(\chi)^2\bnhs,\nonumber\\
 C_\kappa^{\rm obs}(k)&=&\int\!\!\dr\chi~f_\kappa^2P^{\rm
  NL}_m(k;\chi)+\frac{\sigma_\epsilon^2}{\bar{N}_{\rm sg}},
\end{eqnarray}
%
where $\sigma_\epsilon$ is the rms intrinsic ellipticity per component
\cite{TakadaJain:03} and $\bar{N}_{\rm sg}$ is the mean number density
of source galaxies, used for the weak lensing measurements, per unit area
in dimension of $[{\rm Mpc}^{-2}]$, which is given in terms of the
angular source density as $\bar{N}_{\rm sg}=\bar{n}_{\rm
gs}/\ave{\chi_L^2}$, where $\bar{n}_{\rm sg}$ is the mean angular number
density of source galaxies per unit solid steradian as often used in the
literature.  The above formula of the Gaussian term is new in a sense
that the covariance is for the stacked lensing in dimension of the
surface mass density, $\ave{\dsigma}(R)$, while the literature usually
studied the angular power spectrum.

The Kronecker delta function in Eq.~(\ref{eq:cov_g}) ensures that the
Gaussian term contributes only the diagonal terms of the covariance
matrix. The covariance term scales with the survey area as $C^{\rm
Gauss}\propto 1/\Omega_S$.


\subsubsection{Non-Gaussian covariance: the trispectrum consistency and
   super-sample covariance}

Now we consider the non-Gaussian term of the stacked lensing
covariance. As we stated above, we need to include the contributions of
connected 4-point function and the modulation in the projected number
density of lensing halos due to super-survey modes, $\Delta_\br^{\rm h}$
(Eqs.~\ref{eq:def_Deltahb} and \ref{eq:cov_formal}).

To derive different contributions to the non-Gaussian covariance in an
efficient way, we follow the formulation used in
Refs.~\cite{TakadaHu:13,Lietal:14a}. For that purpose, let us consider
the contribution from the connected 4-point function, by inserting the
estimator (Eq.~\ref{eq:e_chm}) into the covariance definition
(Eq.~\ref{eq:def_cov}):
%
\begin{eqnarray}
 \ave{(\hat{h}\hat{\kappa})_{k_i}(\hat{h}\hat{\kappa})_{k_j}}&\leftarrow&
  \frac{\bar{\rho}_{\rm m0}^2}{\bNh(z_L)^2}\int\!\!\dr\chi~f_{\rm h}(\chi)^2
  \int\!\!\dr M_1\int\!\!\dr M_2\frac{\dr n}{\dr M_1}S(M_1)\frac{\dr
  n}{\dr M_2}(M_2)\nonumber\\
 &&\hspace{4em}\times
  \frac{1}{(\chi^2\Omega_S)^2}\int\!\!\frac{\dr^2\bq}{(2\pi)}|\tW(\bq)|^2
  \int_{|\bk|\in k_i}\!\!\frac{\dr^2\bk}{A(k_i)}
  \int_{|\bk'|\in k_j}\!\!\frac{\dr^2\bk'}{A(k_j)}
  T_{\rm hmhm}(\bk,-\bk+\bq,\bk',-\bk'-\bq;M_1,M_2,\chi),
  \label{eq:cov_t}
\end{eqnarray}
%
where we have used the Limber's approximation and $T_{\rm hmhm}$ is the
3D trispectrum at redshift $z=z(\chi)$ for the two matter fields and the
two fields of halos with their masses $M_1$ and $M_2$, defined as
%
\begin{equation}
\ave{\delta_{\rm h}(\bk_1;M_1)\deltam(\bk_2)\delta_{\rm
 h}(\bk_3;M_2)\deltam(\bk_4)}
 \equiv (2\pi)^3\delta_D^3(\bk_{1234})T_{\rm
 hmhm}(\bk_1,\bk_2,\bk_3,\bk_4; M_1,M_2)
\end{equation}
%
The convolution with the window function in Eq.~(\ref{eq:cov_t}) means
that different 4-point configurations separated by less than the Fourier
width of the window function and involving contributions from
super-survey modes contribute to the non-Gaussian covariances. 

For squeezed configurations with $k,k'\gg q$, using the halo model
approach, we can express the change in the trispectrum due to the long
wavelength $q$-mode to leading order in $q/k$ as
%
\begin{equation}
 T_{\rm hmhm}(\bk,-\bk+\bq,\bk',-\bk'-\bq)\equiv T_{\rm
  hmhm}(\bk,-\bk,\bk',-\bk')
  +\delta T(\bk,\bk',q)
  \label{eq:def_dT}
\end{equation}
%
Inserting the first term of the above equation into the covariance
calculation gives the non-Gaussian term arising from the trispectrum of
sub-survey modes:
%
\begin{eqnarray}
 C^{T0}(k_i,k_j)&\equiv &\frac{\bar{\rho}_{\rm m0}^2}{\bNh(z_L)^2}
  \int\!\!\dr\chi~f_{\rm h}(\chi)^2 \bnhs(\chi)^2\frac{1}{\chi^2\Omega_S}
    \bar{T}_{\rm
  hmhm}(k_i,k_j;\chi)
\end{eqnarray}
%
with
%
\begin{equation}
\bar{T}_{\rm hmhm}(k_i,k_j)=\frac{1}{(\bnhs)^2}\int\!\!\dr
 M_1\int\!\!\dr M_2~\frac{\dr n}{\dr M_1}S(M_1)\frac{\dr n}{\dr
 M_2}S(M_2)
 \int^{2\pi}_0\!\!\frac{\dr\phi_{\bk\cdot\bk'}}{2\pi}T_{\rm hmhm}(\bk,-\bk,\bk',-\bk'),
\end{equation}
%
where we have used
$(1/\chi^2\Omega_s)\int\!\!\dr^2\bq/(2\pi)^2|\tW(\bq)|^2=1$, assumed
that the trispectrum is not a rapidly varying function within the
$k,k'$-bin width, and $\phi_{\bk\cdot\bk'}$ is the angle between the
vectors $\bk$ and $\bk'$ in the 2D plane (perpendicular to the line of
sight direction). $C^{T0}$ scales with the survey area as $C^{T0}\propto
1/\Omega_S$ as the Gaussian covariance term does.  In Appendix~XXX we
show expressions from 1 to 4-halo trispectrum terms. 
%employ the halo model to derive expressions for $C^{T0}_{ij}$.


Now we consider the super-sample covariance (SSC) arising from $\delta
T$ in Eq.~(\ref{eq:def_dT}). In doing this we use the halo model that
describes the halo-matter trispectrum involving one to four halos:
%
\begin{equation}
 \delta T^S_{\rm hmhm}
  \equiv \frac{1}{(\bnhs)^2}\int\!\!\dr M_1\int\!\!\dr
  M_2\frac{\dr n}{\dr M_1} S(M_1)\frac{\dr n}{\dr M_2}S(M_2)
  \left[
   T_{\rm hmhm}(\bk,-\bk+\bq,\bk',-\bk'-\bq) -
      T_{\rm hmhm}(\bk,-\bk,\bk',-\bk')
	   \right],
  \label{eq:def_barT}
\end{equation}
%
with
%
\begin{eqnarray}
 &&\delta T^{S,{\rm 1h}}_{\rm hmhm}\approx 0,\nonumber\\
 &&\delta T^{S, {\rm 2h}}_{\rm
  hmhm}\approx \pml(q){\cal I}^{1}_1(k){\cal I}^{1}_1(k'),\nonumber\\
 &&\delta T^{S,{\rm 3h(13)}}_{\rm hmhm}\approx 0,\nonumber\\
 &&\delta T^{S, {\rm 3h(22)}}_{\rm hmhm}\approx 2\pml(q)
\left[{\cal I}^{1}_1(k'){\cal
 F}(\bk,\bq)+{\cal I}^{1}_1(k){\cal F}(\bk',-\bq)\right],\nonumber\\
 &&\delta T^{S, {\rm 4h}}_{\rm hmhm}\approx4\pml(q){\cal
  F}(\bk,\bq){\cal F}(\bk',-\bq),
\end{eqnarray}
%
where
%
\begin{equation}
{\cal F}(\bk,\bq)\equiv
 \left[\pml(k)F_2(\bq,-\bk)+\pml(|\bk-\bq|)F_2(\bq,\bk-\bq)\right]\bar{b}^S{\cal
 I}^{1}_1(|\bk-\bq|).
\end{equation}
%
Here the mode-coupling kernel $F_2$ is
%
\begin{equation}
 F_2(\bk,\bq)\equiv \frac{5}{7}+\frac{1}{2}\left(\frac{1}{k^2}+\frac{1}{q^2}\right)
  (\bk\cdot\bq)+\frac{2}{7}\frac{(\bk\cdot\bq)^2}{k^2q^2}.
\end{equation}
%
Taking into account the fact that the mode coupling factor $F_2$ has a
pole when one of its arguments goes to zero, the integration of ${\cal
F}$ over the direction of $\bk$ is
%
\begin{equation}
 \int_0^{2\pi}\!\!\frac{\dr\phi_\bk}{2\pi}~{\cal F}(\bk,\bq)\simeq
  \frac{1}{2}
  \left(\frac{24}{7}-\frac{1}{2}\frac{d\ln k^2P^{S, {\rm 2h}}_{\rm hm}}{d\ln
   k}\right) P^{S,{\rm 2h}}_{\rm hm}(k)
  \label{eq:def_F}
\end{equation}
%
where we have defined $P^{S,{\rm 2h}}_{\rm hm}(k)=\bar {b}^S\pml(k)$.


Following the trispectrum consistency relation introduced in
Ref.~\cite{TakadaHu:13}, we can realize that the the SSC term in the
squeezed trispectrum arises from the response of the power spectrum to
change in the background density by a factor of $(1+\deltab)$.
Inserting Eq.~(\ref{eq:def_F}) into Eq.~(\ref{eq:def_barT}), we can
rewrite the SSC term as 
%
\begin{eqnarray}
&& \int_0^{2\pi}\!\!\frac{\dr\phi_\bk}{2\pi}
\int_0^{2\pi}\!\!\frac{\dr\phi_{\bk'}}{2\pi}\delta T^S_{\rm hmhm}
 \approx
\pml(q)\frac{\partial \phm^S(k)}{\partial \deltab}\frac{\partial \phm^S(k')}{\partial \deltab}
\end{eqnarray}
%
where we have defined the response of the halo-matter power spectrum to
the background density:
%
\begin{equation}
 \frac{\partial \phm^S(k)}{\partial \deltab}\equiv 
  \left(\frac{24}{7}-\frac{1}{2}\frac{\dr\ln k^2\phm^{S,{\rm
   2h}}(k)}{\dr \ln k}\right)\phm^{S,{\rm 2h}}(k) + {\cal I}^1_1(k).
\end{equation}
%
Comparing the above equation with Eq.~(27) in Ref.~\cite{Lietal:14a},
which is the response of the 3D mass power spectrum, the differences,
$24/7$ instead of $68/21$ and $(1/2)\dr\ln k^2P(k)/\dr \ln k$ instead of
$(1/3)\dr\ln k^3P(k)/\dr \ln k$, are due to the fact that we consider
the projected power spectrum and the super-survey mode of $\bq$ in the
2D plan perpendicular to the line-of-sight direction; that is, we
ignored the radial mode.  The $24/7$ piece is called the beat coupling
(``BC'') effect \cite{Hamiltonetal:06} that the growth of a short
wavelength perturbation is enhanced in a large scale overdensity, the
${\cal I}^1_1$ term is the halo sample variance (``HSV'') effect
\cite{TakadaBridle:07,Satoetal:09} that halo number densities are also
enhanced in such a region, and the derivative term is the linear
dilation (``LD'') effect. The dilation effect occurs because the long
wavelength mode changes the local expansion factor and hence the
physical size of a mode that would have comoving wavelength $k$ in its
absence.


Furthermore, including a modulation in the projected number density
halos, $\hNh$ (Eq.~\ref{eq:def_Deltahb}), we can realize that the
background mode $\deltab$ causes a modulation in the power spectrum
estimator of stacked lensing, to the linear order of $\deltab$:
%
\begin{equation}
 \left.\delta\hchm(k)\right|_{\deltab}=
  \frac{\bar{\rho}_{\rm m0}}{\bNh(z_L)}\int\!\!\dr\chi~f_{\rm h}(\chi)\bnhs
  \frac{\partial \phm^S(k)}{\partial
  \deltab}\deltab-\frac{\chm(k)}{\bNh(z_L)}\int\!\!\dr\chi~f_{\rm h}(\chi)\bnhs\bar{b}^S\deltab.
\label{eq:dchm_ddb_prep}
\end{equation}
%
From Eqs.~(\ref{eq:phmS}) and (\ref{eq:dhnhS_ddb}), we can redefine the
halo-matter power spectrum $\phm^S$ so that it incldues a modulation in
the mean 3D number density of halos due to the background mode as
%
\begin{equation}
 \hphm^S(k)=\frac{1}{\hnh^S(\chi)}\int\!\!\dr~M \bnh S(M)\phm(k;M,\chi)
\end{equation}
%
then we can find
%
\begin{equation}
 \frac{\partial \hphm^S(k)}{\partial \deltab}\simeq
  \frac{1}{\bnhs(\chi)}\int\!\!\dr~M \bnh S(M)
  \frac{\partial \phm(k;M,\chi)}{\partial \deltab}-\bar{b}^S \phm. 
\end{equation}
%
Hence Eq.~(\ref{eq:dchm_ddb_prep}) can be rewritten by a simpler form as
%
\begin{equation}
 \left.\delta \chm(k)\right|_{\deltab}=\frac{\bar{\rho}_{\rm
  m0}}{\bNh(z_L)}
  \int\!\!\dr\chi~f_{\rm h}(\chi)\bnhs(\chi)\frac{\partial
  \hphm^S(k)}{\partial \deltab}\deltab
\end{equation}


%where $\delta|_{\deltab}$ denotes a variation in the estimator $\hchm$
%due to a change in the background mode.
For a thin redshift slice of lensing halos, the power spectrum response
can be simplified as
%
\begin{equation}
 \left.\delta\hchm(k)\right|_{\deltab}\simeq \bar{\rho}_{\rm m0}
  \left[\left(\frac{24}{7}-\frac{1}{2}\frac{\dr\ln k^2\phm^{S,{\rm
	 2h}}}{\dr \ln k}\right)+{\cal I}^1_1-\bar{b}^S\phm^S(k)\right]\deltab
\end{equation}
%
This corresponds to a response of the projected power spectrum for the
local mean density in Ref.~\cite{TakadaHu:13}.

\begin{equation}
 \left.\delta\hchm(k)\right|_{\deltab}=\frac{\bar{\rho}_{\rm m0}}{\bNh(z_L)}
  \int\!\!\dr\chi~f_{\rm h}(\chi)\bnhs
\frac{\partial \hphm^S(k)}{\partial \deltab}\deltab
\end{equation}

\begin{equation}
 \hphm^S(k)\equiv \frac{1}{\hnh^S(\chi)}\int\!\!\dr M~\bnh S(M)\phm(k;M,\chi)
\end{equation}
%
\begin{eqnarray}
\frac{\partial \hphm^S(k)}{\partial \deltab}&=&\frac{1}{\bnhs}\int\!\!\dr
 M\!\bnh S(M)\frac{\partial \phm^S}{\partial  \deltab}
-\phm^S\frac{\partial \ln \hnh^S}{\partial \deltab}
\end{eqnarray}
%
\begin{equation}
 \hnh^S \equiv \int\!\!\dr M~ \bnh S(M)\left(1+b\deltab\right)
\end{equation}
%
\begin{equation}
 \frac{\partial \hnh^S}{\partial \deltab}\equiv \int\!\!\dr M~ \bnh
  S(M)b = \bnhs \bar{b}^S
\end{equation}



Hence, using the Limber's approximation, we can compute the SSC term of
the non-Gaussian covariance as
%
\begin{eqnarray}
 C^{\rm SSC}(k_i,k_j)&\equiv& \ave{\delta\hchm(k_i)|_{\delta
  b}\delta\hchm(k_j)|_{\deltab}}\nonumber\\
  &\simeq &\frac{1}{\bNh(z_L)^2}
  \int\!\!\dr\chi~f_{\rm h}(\chi)^2(\bnhs)^2\left[
  \bar{\rho}_{\rm m0}^2\frac{\partial \phm^S(k_i)}{\partial \deltab}
  \frac{\partial \phm^S(k_j)}{\partial \deltab}
  -\bar{\rho}_{\rm m0}\bar{b}^S
  \chm(k_i)\frac{\partial \phm^S(k_j)}{\partial
  \deltab}\right.\nonumber\\
 &&\hspace{8em}\left.
-\bar{\rho}_{\rm m0}\bar{b}^S\chm(k_j)\frac{\partial \phm^S(k_i)}{\partial \deltab}
+(\bar{b}^S)^2\chm(k_i)\chm(k_j)
		\right]\frac{1}{(\chi^2\Omega_S)^2}\int\!\!\frac{\dr^2\bq}{(2\pi)^2}|\tW(\bq)|^2\pml(q;\chi).
\end{eqnarray}
%
\cite{TakadaSpergel:13,Schaanetal:14}

\begin{equation}
\Delta\sigma_b^2(\chi)\equiv \int\!\!d\chi~
\end{equation}

\subsection{Super sample signal}

\begin{equation}
\Delta_{\rm b}\equiv \frac{1}{\bNh}\int\!\!\dr\chi~f_{\rm
 h}(\chi)\bnhs(\chi)\frac{1}{\chi^2\Omega_S}
 \int\!\!\dr^2\bx_\perp~ W(\bx_\perp)\deltab(\bx;\chi)
\end{equation}
%


\subsubsection{Summary: a halo model formula for the covariance of
   stacked lensing power spectrum}

\mtrv{MT: up to here so far}
   
\section{Results}



\section{Discussion}

%\bigskip

\smallskip{\em Acknowledgments.--} We thank Wayne Hu, Yin Li and Surhud
More for useful discussion. MT is supported in part by Grant-in-Aid for
Scientific Research from the JSPS Promotion of Science (No. 23340061 and
26610058), MEXT Grant-in-Aid for Scientific Research on Innovative Areas
(No. 15H05893) and by JSPS Program for Advancing Strategic International
Networks to Accelerate the Circulation of Talented Researchers.



\bibliography{sss_Phm}

\appendix

\section{An estimator of the stacked lensing profile}


\begin{eqnarray}
 w_{\mathrm{hh}}(R;z_L)&=&\ave{\delta^{\rm 2D}_\bh(\bx_\perp;z_L)
  \delta^{\rm 2D}_\bh(\bx'_\perp;z_L)
  }_{R=|\bx_\perp-\bx'_\perp|}\nonumber\\
 &=&\frac{1}{\bar{N}_\bh(z_L)^2}\int\!\dr\chi \chi^2
  \int\!\dr\chi \chi^{\prime 2}
  \int\!\dr M\frac{\dr n}{\dr M}S(M, \chi; z_L)
  \int\!\dr M'\frac{\dr n}{\dr M'}S(M',\chi'; z_L)\nonumber\\
  &&\hspace{2em}\times
  \int\!\frac{\dr k_\parallel \dr^2\bk_\perp}{(2\pi)^3}
  \int\!\frac{\dr k'_\parallel \dr^2\bk'_\perp}{(2\pi)^3}
  \ave{\deltah(\bk;\chi, M)\deltah(\bk';\chi', M')
  }e^{i(\chi k_\parallel+\bx_\perp\cdot\bk_\perp)
  +i(\chi'k'_\parallel+\bx'_\perp\cdot\bk'_\perp)}\nonumber\\
 &=&\frac{1}{\bar{N}_\bh(z_L)^2}\int\!\dr\chi \chi^2
  \int\!\dr\chi \chi^{\prime 2}
  \int\!\dr M\frac{\dr n}{\dr M}S(M, \chi; z_L)
  \int\!\dr M'\frac{\dr n}{\dr M'}S(M',\chi'; z_L)\nonumber\\
  &&\hspace{4em}\times
  \int\!\frac{\dr k_\parallel \dr^2\bk_\perp}{(2\pi)^3}
  \phh(k;M,M')
  e^{ik_\parallel(\chi-\chi')+i\bk_\perp\cdot{\bf R}}\nonumber\\
 &\simeq& \frac{1}{\bar{N}_\bh(z_L)^2}\int\!\dr\chi \chi^4
  \int\!\dr M\int\!\dr M'~\frac{\dr n}{\dr M}S(M, \chi; z_L)
  \frac{\dr n}{\dr M'}S(M',\chi'; z_L)
\int\!\frac{\dr^2\bk_\perp}{(2\pi)^2}
\phh(k_\perp;M,M',z)
  e^{i\bk_\perp\cdot{\bf R}}\nonumber\\
 &\simeq& \frac{1}{\bar{N}_\bh(z_L)^2}\chi_L^4\Delta\chi
  \int\!\dr M\int\!\dr M'~\frac{\dr n}{\dr M}S(M, \chi; z_L)
  \frac{\dr n}{\dr M'}S(M',\chi'; z_L)
\int\!\frac{\dr^2\bk_\perp}{(2\pi)^2}
\phh(k_\perp;M,M',z)
e^{i\bk_\perp\cdot{\bf R}}
%\nonumber\\
% &=& \frac{1}{\bnh(z_L)^2\Delta\chi}
%  \int\!\dr M\int\!\dr M'~\frac{\dr n}{\dr M}S(M, \chi; z_L)
%  \frac{\dr n}{\dr M'}S(M',\chi'; z_L)
%\int\!\frac{\dr^2\bk_\perp}{(2\pi)^2}
%\phh(k_\perp;M,M',z)
%  e^{i\bk_\perp\cdot{\bf R}}\nonumber\\
\end{eqnarray}



Using the Limber's approximation, the ensemble average of the sacked
lensing profile can be computed as
%
\begin{eqnarray}
 \ave{\Sigma}(R)&\equiv&\ave{\sigmacr(\chi)\kappa^W(\btheta)}_{R=|\bx_\perp(z_l)-\chi(z_l)\btheta|}
=\ave{\deltah(\bx_\perp)\sigmacr(\chi)\kappa^W(\bx'_\perp)}_{R=|\bx_\perp-\bx'_\perp|;\bx'=\chi\btheta}
\nonumber\\
  &=&
  \frac{1}{\hNh(z_L)}\int\!\dr\chi \dr\chi'~\chi^2 f_\bh(\chi)\sigmacr(\chi)f_\kappa(\chi')
  \int\!\dr M~\bnh S(M)b(M)W(\bx_\perp)\ave{\deltaml(\chi,\bx_\perp)\deltam^W(\chi',\chi'\btheta)}
  \nonumber\\
&=& \frac{1}{\hat{N}_\bh(z_L)}\int\!\dr\chi \dr\chi'~\chi^2\sigmacr(\chi)W_\kappa(\chi')
 \int\!\dr M~\frac{\dr n}{\dr M}S(M,\chi;z_L)\nonumber\\
&&\hspace{6em}\times \int\!\frac{d^3\bk}{(2\pi)^3}\frac{d^3\bk'}{(2\pi)^3}
\phm(k;M,\chi)(2\pi)^3\delta_D(k_\parallel+k'_\parallel)\delta_D^2(\bk_\perp+\bk'_\perp)
e^{ik_\parallel\chi+i\bk_\perp\cdot\bx_\perp}
e^{ik_\parallel\chi'+i\bk'_\perp\cdot\chi\btheta}
  \nonumber\\ 
&=& \frac{1}{\hat{N}_\bh(z_L)}\int\!\dr\chi \dr\chi'~\chi^2\sigmacr(\chi)W_\kappa(\chi')
 \int\!\dr M~\frac{\dr n}{\dr M}S(M,\chi;z_L)
\int\!\frac{\dr k_\parallel \dr^2\bk_\perp}{(2\pi)^3}
\phm(k;M,\chi)
e^{ik_\parallel(\chi-\chi')+i\bk_\perp\cdot(\bx_\perp-\chi'\btheta)}
  \nonumber\\ 
&\simeq & \frac{1}{\hat{N}_\bh(z_L)}\int\!\dr\chi \dr\chi'~\chi^2\sigmacr(\chi)W_\kappa(\chi')
 \int\!\dr M~\frac{\dr n}{\dr M}S(M,\chi;z_L)
\int\!\frac{\dr k_\parallel \dr^2\bk_\perp}{(2\pi)^3}
\phm(k_\perp;M,\chi)
e^{ik_\parallel(\chi-\chi')+i\bk_\perp\cdot(\bx_\perp-\chi'\btheta)}
  \nonumber\\ 
&= & \frac{1}{\hat{N}_\bh(z_L)}\int_{\chi_L}\!\dr\chi ~\chi^2\sigmacr(\chi)W_\kappa(\chi)
 \int\!\dr M~\frac{\dr n}{\dr M}\tilde{S}(M)
\int\!\frac{\dr^2\bk_\perp}{(2\pi)^2}
\phm(k_\perp;M,\chi)
e^{i\bk_\perp\cdot(\bx_\perp-\chi\btheta)}
\nonumber\\
&=&\frac{\bar{\rho}_{\rm m0}}{\hat{N}_\bh(z_L)}\int_{\chi_L}\!\dr\chi ~\chi^2
 \int\!\dr M~\frac{\dr n}{\dr M}\tilde{S}(M)
\int\!\frac{\dr^2\bk_\perp}{(2\pi)^2}
\phm(k_\perp;M,\chi)
e^{i\bk_\perp\cdot(\bx_\perp-\chi\btheta)},
\end{eqnarray}
%
where $\phm(k;M,z)$ is the halo-mass cross-power spectrum for halos of
mass $M$ and at redshift $z$.
%Thus cross-correlation of background
%galaxy with foreground lens halos of known redsift arises from an
%integration of the halo-mass cross-power spectrum at each lens
%redshift.

\section{Halo model for the non-Gaussian covariance term $C^{T0}$}

\begin{equation}
 C^{T0}(k_i,k_j)=\frac{\bar{\rho}_{\rm m0}^2}{\bNh(z_L)^2}
  \int\!\!\dr\chi~f_{\rm h}(\chi)^2(\bnhs)^2\bar{T}_{\rm hmhm}(k_i,k_j),
\end{equation}
%
\begin{equation}
 T^S_{\rm hmhm}(\bk,\bk')=\frac{1}{(\bnhs)^2}
  \int\!\!\dr M_1\int\!\!\dr M_2 \frac{\dr n}{\dr M_1}S(M_1)
  \frac{\dr n}{\dr M_2}S(M_2)T_{\rm hmhm}(\bk,-\bk,\bk',-\bk')
\end{equation}
%
\begin{equation}
 T_{\rm hmhm}=\bar{T}_{\rm hmhm}^{\rm 1h}
  +\left(\bar{T}_{\rm hmhm}^{\rm 2h(13)}
    +\bar{T}_{\rm hmhm}^{\rm 2h(22)}
	  \right)
   +\bar{T}_{\rm hmhm}^{\rm 3h}
   +\bar{T}_{\rm hmhm}^{\rm 4h}
\end{equation}
%
\begin{eqnarray}
 \bthmhm^{S, {\rm 1h}}(\bk,\bk')&=&{\cal I}^{4}_0(0,k,0,k')\nonumber\\
% 1-(234) 2-(134) 3-(124) 4-(123)
 \bthmhm^{S, {\rm 2h(13)}}(\bk,\bk')&=&\bar{b}^S{\cal
 I}^1_{2}(k,k')\pml(k)
 +I^1_1(k){\cal I}^1_1(k')\pml(k)
 +\bar{b}^S{\cal I}^1_2(k,k')\pml(k')
 +I^1_1(k'){\cal I}^1_1(k)\pml(k'),\nonumber\\
%(12)(34) (13)(24) (14)(23)
 \bthmhm^{S, {\rm 2h(22)}}(k_i,k_j)&=&\bar{b}^SI^1_2(k_i,k_j)\pml(|\bk+\bk'|)
 +{\cal I}^1_1(k_i){\cal I}^1_1(k_j)\pml(|\bk-\bk'|)\nonumber\\
 %(12)34 (13)24 (14)23 (23)14 (24)13 (34)12
 \bthmhm^{S, {\rm 3h}}(\bk,\bk')&=&
 \bar{b}^SI^1_1(k)I^1_1(k')B^{\rm PT}_{\rm m}(\bk+\bk',-\bk,-\bk')
 +\bar{b}^S{\cal I}^1_1(k')I^1_1(k)B^{\rm PT}_{\rm m}(\bk-\bk',-\bk,\bk')\nonumber\\
&& +\bar{b}^S{\cal I}^1_1(k)I^1_1(k')B^{\rm PT}_{\rm m}(-\bk+\bk',\bk,-\bk')
 +\bar{b}^S{\cal I}^1_1(k)I^1_1(k')B^{\rm PT}_{\rm
 m}(-\bk-\bk',\bk,\bk')\nonumber\\
 \bthmhm^{S, {\rm 4h}}(\bk,\bk')&=&(\bar{b}^S)^2I^1_1(k)I^1_1(k')T^{\rm PT}_m(\bk,-\bk,\bk',-\bk')
\end{eqnarray}
\begin{equation}
 \bar{T}_{\rm hmhm}=\bar{T}_{\rm hmhm}^{\rm 1h}
  +\left(\bar{T}_{\rm hmhm}^{\rm 2h(13)}
    +\bar{T}_{\rm hmhm}^{\rm 2h(22)}
	  \right)
   +\bar{T}_{\rm hmhm}^{\rm 3h}
   +\bar{T}_{\rm hmhm}^{\rm 4h}
\end{equation}
%
\begin{eqnarray}
 \bthmhm^{\rm 1h}(\bk,\bk')&=&{\cal I}^{4}_0(0,k_i,0,k_j)\nonumber\\
% 1-(234) 2-(134) 3-(124) 4-(123)
 \bthmhm^{\rm 2h(13)}(k_i,k_j)&=&\bar{b}^S{\cal
 I}^1_{3}(0,k_i,k_j)\pml(k_i)
 +I^1_1(k_i){\cal I}^1_3(0,0,k_j)\pml(k_i)
 +\bar{b}^S{\cal I}^1_3(0,k_i,k_j)\pml(k_j)
 +I^1_1(k_j){\cal I}^1_3(0,0,k_i)\pml(k_j),\nonumber\\
%(12)(34) (13)(24) (14)(23)
 \bthmhm^{\rm 2h(22)}(k_i,k_j)&=&\bar{b}^SI^1_2(k_i,k_j)\pml(|\bk+\bk'|)
 +{\cal I}^1_1(k_i){\cal I}^1_1(k_j)\pml(|\bk-\bk'|)\nonumber\\
 %(12)34 (13)24 (14)23 (23)14 (24)13 (34)12
 \bthmhm^{\rm 3h}(\bk,\bk')&=&
 \bar{b}^SI^1_1(k)I^1_1(k')B^{\rm PT}_{\rm m}(\bk+\bk',-\bk,-\bk')
 +\bar{b}^S{\cal I}^1_1(k')I^1_1(k)B^{\rm PT}_{\rm m}(\bk-\bk',-\bk,\bk')\nonumber\\
&& +\bar{b}^S{\cal I}^1_1(k)I^1_1(k')B^{\rm PT}_{\rm m}(-\bk+\bk',\bk,-\bk')
 +\bar{b}^S{\cal I}^1_1(k)I^1_1(k')B^{\rm PT}_{\rm
 m}(-\bk-\bk',\bk,\bk')\nonumber\\
 \bthmhm^{\rm 4h}(\bk,\bk')&=&(\bar{b}^S)^2I^1_1(k)I^1_1(k')T^{\rm PT}_m(\bk,-\bk,\bk',-\bk')
\end{eqnarray}

\begin{eqnarray}
 \ave{\Sigma}(R;z_L)&\equiv&
  \ave{\deltah(\bx_\perp;z_L)\sigmacr(z_L)\kappa(\btheta)}_{|\bx_\perp-\chi_L
  \btheta|=R}\nonumber\\
 &=&\frac{1}{\bar{N}_\bh(z_L)}\int\!\dr\chi~\chi^2\int\!\dr\chi'~f_\kappa(\chi)\sigmacr(\chi)
  \int\!\dr M~\frac{\dr n}{\dr M}S(M,\chi;z_L)
  \ave{\deltah(\chi,\bx_\perp;M)\deltam(\chi',\chi'\btheta)}\nonumber\\
 &=&\frac{1}{\bar{N}_\bh(z_L)}\int\!\dr\chi~\chi^2\int\!\dr\chi'~f_\kappa(z)\sigmacr(\chi)
  \int\!\dr M~\frac{\dr n}{\dr M}S(M,\chi;z_L)\nonumber\\
 &&\hspace{4em}\times
  \int\!\frac{\dr^2\bk_\perp \dr k_\parallel}{(2\pi)^3}
  \int\!\frac{\dr^2\bk'_\perp \dr k'_\parallel}{(2\pi)^3}
  \ave{\deltah(\bk;M)\deltam(\bk')}
e^{i(\chi k_\parallel+\bk_\perp\cdot\bx_\perp)+i(\chi'k'_\parallel+\bk'_\perp\cdot\chi'\btheta)}
\nonumber\\
 &=&\frac{1}{\bar{N}_\bh(z_L)}\sigmacr(z_L)\int\!\dr\chi~\chi^2\int\!\dr\chi'~f_\kappa(z)
  \int\!\dr M~\frac{\dr n}{\dr M}S(M,\chi;z_L)
  \int\!\frac{\dr^2\bk_\perp \dr k_\parallel}{(2\pi)^3}
\phm(k;M)
e^{ik_\parallel(\chi-\chi')+i\bk_\perp\cdot(\bx_\perp-\chi'\btheta)}
\nonumber\\
 &\simeq&
  \frac{1}{\bar{N}_\bh(z_L)}\sigmacr(z_L)\int\!\dr\chi~\chi^2f_\kappa(\chi)
  \int\!\dr M~\frac{\dr n}{\dr M}S(M,\chi;z_L)
  \int\!\frac{\dr^2\bk_\perp}{(2\pi)^2}
\phm(k_\perp;M)
e^{i\bk_\perp\cdot(\bx_\perp-\chi\btheta)}
\nonumber\\
 &\simeq&   \frac{1}{\bar{N}_\bh(z_L)}\sigmacr(z_L)\Delta\chi
\chi_L^2f_\kappa(\chi_L)
  \int\!\dr M~\frac{\dr n}{\dr M}S(M;z_L)
  \int\!\frac{\dr^2\bk_\perp }{(2\pi)^2}
\phm(k_\perp;M)
e^{i\bk_\perp\cdot(\bx_\perp-\chi_L\btheta)}\nonumber\\
 &=&\frac{1}{\bnh(z_L)}\bar{\rho}_{\rm m0}
  \int\!\dr M~\frac{\dr n}{\dr M}S(M;z_L)
  \int\!\frac{\dr^2\bk_\perp }{(2\pi)^2}
\phm(k_\perp;M)
e^{i\bk_\perp\cdot(\bx_\perp-\chi_L\btheta)}
\end{eqnarray}
%

%
\begin{equation}
 {\bf C}={\bf C}^{\rm Gauss}+{\bf C}^{T}
\end{equation}
%
\begin{equation}
 C^{\rm Gauss}_{ij}=\frac{2}{N_{\rm mode}(k_i)}\delta^K_{ij}
  \left[\chm(k_i)^2+C_{\rm hh}^{\sigmacr}(k_i)C_\kappa(k_i)\right]
\end{equation}
%
\begin{eqnarray}
N_{\rm mode}(k)&\simeq &\int\!\!\dr\chi~\, f_\bh(\chi)\frac{2\pi k\Delta
 k}{(2\pi/\chi\Theta_S)^2}=\Omega_S
 \int\!\!\dr\chi~\, f_\bh(\chi)\frac{ k\Delta k\chi^2}{2\pi}\simeq
 \frac{k\Delta k\chi_L^2\Omega_S}{2\pi}\nonumber\\
 \chh^{\sigmacr}(k)&=&\frac{1}{\bNh(z_L)^2}
  \int\!\!\dr\chi\, f_\bh(\chi)^2\sigmacr(\chi)^2\left[\int\!\!\dr
					    M\bnh b(M)\right]^2\pml(k;\chi)+
  \frac{\int\!\!\dr\chi\, f_\bh(\chi)\sigmacr(\chi)^2\int\!\!\dr M\,
  \bnh}{\bNh(z_L)^2}\nonumber\\
C_\kappa(k)&=&\int\!\!\dr\chi\,f_\kappa(\chi)^2P^{\rm NL}_{\rm
 m}(k;\chi)+\frac{\sigma_\epsilon^2}{\bar{n}_{gs}},
\end{eqnarray}
%

Using Limeber's approximation,
%
\begin{equation}
C^T_{ij}=\frac{1}{\bNh(z_L)^2}\int\!\!\dr\chi\,
 f_\bh(\chi)^2\sigmacr(\chi)^2
 \frac{1}{(\chi^2\Omega_S)^2}\int\!\frac{\dr^2\bq}{(2\pi)^2}|\tW(\bq)|^2
 \int_{k_i}\!\!\frac{\dr\bk}{V(k_i)}
 \int_{k_j}\!\!\frac{\dr\bk}{V(k_j)}
 T_{\rm hmhm}(\bk,-\bk+\bq,\bk',-\bk'-\bq)
\end{equation}

For squeezed configurations with $k,k'\gg q$ we can express the change
in the trispectrum due to the long wavelength $q$ mode to leading order
in $q/k$ as
%
\begin{equation}
\delta T_{\rm hmhm}\equiv T_{\rm
 hmhm}(\bk,-\bk+\bq,\bk',-\bk'-\bq)-T_{\rm hmhm}(\bk,-\bk,\bk',-\bk')
\end{equation}
%
with
%
\begin{eqnarray}
 &&\delta T^{\rm 1h}_{\rm hmhm}\simeq 0\nonumber\\
 &&\delta T^{\rm 2h(22)}_{\rm hmhm}\simeq
  \pml(q)I^{(S)1}_1(k)I^{(S)1}_1(k')\nonumber\\
 &&\delta T^{\rm 2h(13)}_{\rm hmhm}\simeq 0\nonumber\\
 &&\delta T^{\rm 3h}_{\rm hmhm}\simeq 2\pml(q)I_1^{(S)1}(k'){\cal
  F}(\bk,\bq)+2\pml(q)I_1^{(S)1}(k){\cal F}(\bk',-\bq)\nonumber\\
 &&\delta T^{\rm 4h}_{\rm hmhm}\simeq 4\pml(q){\cal F}(\bk,\bq){\cal F}(\bk',-\bq)
\end{eqnarray}
%
where
%
\begin{eqnarray}
 {\cal F}(\bk,\bq)\equiv\left[
\pml(k)F_2(\bq,-\bk)+\pml(|\bk-\bq|)F_2(\bq,\bk-\bq)
			\right]I^{(S)0}_0I_1^{(S)1}(\bk-\bq)
\end{eqnarray}
%
with the mode-coupling kernel $F_2$ defined as
%
\begin{equation}
 F_2(\bk,\bq)\equiv \frac{5}{7}+\frac{1}{2}\left(\frac{1}{k^2}+\frac{1}{q^2}\right)(\bk\cdot\bq)
+\frac{2}{7}\frac{(\bk\cdot\bq)^2}{k^2q^2}
\end{equation}
%
The latter must be handled with care since the mode coupling factor
$F_2$ has a pole when one of its arguments goes to zero. Thus we need to
consistently expand this expression in $q/k$ (or $q/k'$).  The result of
integrating over the direction of $\bk$ is
%
\begin{equation}
 \int^{2\pi}_0\!\!\frac{\dr\phi}{2\pi}{\cal F}(\bk,\bq)=\frac{1}{2}
\left[\frac{24}{7}-\frac{1}{2}\frac{\dr\ln \phml(k)}{\dr \ln
 k}\right]\phml(k)
\end{equation}

\begin{equation}
 \frac{\dr \ln P_{\rm hm}(k)}{\dr \deltab}=
  \left[\frac{24}{7}-\frac{1}{2}\frac{\dr\ln \phml(k)}{\dr \ln
 k}\right]\phml(k)+I^{(S)1}_1(k)
\end{equation}


\begin{eqnarray}
 {\bf C}^{\rm SSC}&\leftarrow &\left\langle
  \left[1-2
\int\!\!\dr\chi~f_{\rm h}(\chi)\int\!\!\dr MS(M)\bnh b\deltab
  \right]
  \left[
   \int\!\!\dr\chi~f_{\rm h}(\chi)P^{S}_{\rm hm}(k)\left\{
1+
\frac{\dr\ln P^{S}_{\rm hm}(k)}{\dr \deltab}\deltab
				  \right\}
  \right]\right.\nonumber\\
&&\hspace{12em}\times \left.  \left[
   \int\!\!\dr\chi~f_{\rm h}(\chi)P^{S}_{\rm hm}(k)\left\{
1+
\frac{\dr\ln P^{S}_{\rm hm}(k')}{\dr \deltab}
\deltab\right\}
				 \right]\right\rangle\nonumber\\
 &\simeq &
\frac{1}{\bNh^2}
\int \!\!\dr\chi~f_{\rm h}(\chi)^2
\left[
\frac{\dr\ln P^{S}_{\rm hm}(k)}{\dr \deltab}
\frac{\dr\ln P^{S}_{\rm hm}(k')}{\dr \deltab}
-2I^{(S)1}_0
\frac{\dr\ln P^{S}_{\rm hm}(k)}{\dr \deltab}
-2I^{(S)1}_0
\frac{\dr\ln P^{S}_{\rm hm}(k')}{\dr \deltab}
\right]\frac{1}{(\chi^2\Omega_S)^2}\int\!\!\frac{\dr^2\bq}{(2\pi)}|\tW(\bq)|^2\pml(q)
\end{eqnarray}

First of all, note that we define the halo number density fluctuation
relative to the mean within the survey window, as shown from
Eq.~(\ref{eq:hNh}): 
%
\begin{equation}
 \hchm(k_i)=\frac{\chm(k_i)}{1+\delta \ln\hNh|_{\deltab}}\simeq
  \chm(k_i)
  \left[1-\delta\ln\hNh|_{\deltab}\right]
\end{equation}

\begin{eqnarray}
 \delta\ln\hchm(k_i)|_{\deltab}\simeq \frac{1}{\bNh(z_L)}\int\!\dr \chi\,
  f_\bh(\chi)\int\!\!\dr M\, \bnh(M)S(M)b(M)\deltab(\chi)
\end{eqnarray}



\begin{eqnarray}
 \ave{ \delta\ln\hchm(k_i)|_{\deltab} \delta\ln\hchm(k_j)|_{\deltab}}
  &=&\frac{1}{\bNh(z_L)^2}
  \int\!\!\dr\chi\int\!\!\dr\chi'\, f_\bh(\chi)f_\bh(\chi')
  \int\!\!\dr M\int\!\!\dr M'\, S(M)\bnh(M)S(M)\bnh(M')
  b(M)b(M')\ave{\deltab(\chi)\deltab(\chi')}\nonumber\\
 &&\hspace{-10em}=\frac{1}{\bNh(z_L)^2}
 \int\!\!\dr\chi\int\!\!\dr\chi'\, f_\bh(\chi)f_\bh(\chi')
  \int\!\!\dr M\int\!\!\dr M'\, S(M)\bnh(M)S(M')\bnh(M')
  b(M)b(M')
\nonumber\\
 &&\hspace{-8em}\times
  \frac{1}{\chi^2\Omega_S\chi^{\prime 2}\Omega_S}
\int\!\!\frac{\dr k_\parallel}{2\pi}\int\!\!\frac{\dr k'_\parallel}{2\pi}
\int\!\! \frac{\dr^2 \bq}{(2\pi)^2}
\int\!\! \frac{\dr^2 \bq'}{(2\pi)^2}
\tW(\bq)tW(\bq')\ave{\deltaml(k_\parallel,-\bq)\deltaml(k'_\parallel,-\bq')}
e^{ik_\parallel\chi+i\chi'k'_\parallel}\nonumber\\
 &&\hspace{-10em}=\frac{1}{\bNh(z_L)^2}
\int\!\!\dr\chi\int\!\!\dr\chi'\, f_\bh(\chi)f_\bh(\chi')
  \int\!\!\dr M\int\!\!\dr M'\, S(M)\bnh(M)S(M')\bnh(M')\nonumber\\
&&\hspace{-8em}\times  \frac{1}{\chi^2\Omega_S\chi^{\prime 2}\Omega_S}
\int\!\!\frac{\dr k_\parallel}{2\pi}
\int\!\! \frac{\dr^2 \bq}{(2\pi)^2}
|\tW(\bq)|^2\pml\!\left(\sqrt{k_\parallel^2+\bq^2}\right)
e^{ik_\parallel(\chi-\chi')}\nonumber\\
 &&\hspace{-10em}\simeq \frac{1}{\bNh(z_L)^2}\int\!\!\dr\chi\,
  f_\bh(\chi)^2\left[\int\!\!\dr M\, S(M)\bnh(M)b(M)\right]^2
  \frac{1}{(\chi^2\Omega_S)^2}
\int\!\! \frac{\dr^2 \bq}{(2\pi)^2}
|\tW(\bq)|^2\pml\!(q)
\end{eqnarray}

The covariance matrix describes expected statistical
errors in measurements of the stacked lesing profile as well as how the
stacked lensing power spectra of different wavenumber bins are correted
with each other. The covariance matrix has three contributions: the
Gaussian term, the non-Gaussian term arising from sub-volume modes, and
the non-Gaussian, super-sample covariance (SSC) term arising from
super-survey modes.
%
\begin{eqnarray}
 {\rm Cov}[P_{\dsigma}(k),P_{\dsigma}(k')]&=&{\bf C}^{\rm Gauss}
  +{\bf C}^{\rm sub-NG}
  +{\bf C}^{\rm SSC}\nonumber\\
  &&\hspace{-3em}=
  \frac{\delta_{kk'}^K}{N_{\rm mode}(k)}
\left[P_{\dsigma}(k)^2
 +P^{\rm obs}_{\mathrm{hh}}(k)\sigmacr(z_L)^2\chi_L^2C_{\kappa}^{\rm
 obs}(l=\chi_L k)\right]
%\nonumber\\
%&&
 +\frac{1}{\chi_L^2\Omega_{\rm s}}\bar{T}^{\rm
 2D}_{\mathrm{hhmm}}(k,k';z_L)
 +\sigma_b^2\frac{\partial P_{\dsigma}(k)}{\partial \delta_b}
 \frac{\partial P_{\dsigma}(k')}{\partial \delta_b}
\end{eqnarray}
%
where $N_{\rm mode}(k)$ is the number of Fourier modes taken from the
survey volume, $\phh(k)$ is the auto-power spectrum of halos, and
$C_\kappa(l)$ is the cosmic shear power spectrum.
%
\begin{eqnarray}
 N_{\rm mode}(k)&\equiv& \frac{2\pi k\Delta
  k}{[2\pi/(\chi_L\Theta_s)]^2}=k\Delta k~ \chi_L^2\Omega_{\rm
  s},\nonumber\\
 \phh^{\rm obs}(k)&=&\phh(k)+\frac{1}{\bar{N}_\bh(z_L)}\nonumber\\
 C^{\rm obs}_\kappa(l)&=&C_{\kappa}+\frac{\sigma_\epsilon^2}{\bar{n}_{sg}}.
\end{eqnarray}
%
The halo auto-power spectrum term accounts for the sample varinace
arising due to a finite number of halos in the survey region. The cosmic
shear term accounts for the contribution arising from the mass
distribution along the line-of-sight to source redshifts, and can be
computed in terms of the nonlinear mass power spectrum:
%
\begin{equation}
C_\kappa(l)=\int_{\chi\neq
 [\chi_L-\Delta\chi/2,\chi_L+\Delta/2]}\!\dr\chi~W_\kappa(\chi)^2\chi^{-2}
 P^{\rm NL}_{\rm m}\!\left(k=\frac{l}{\chi}; \chi\right), 
\end{equation}
%
where the line-of-sight integration excludes the contribution of lens
redshifts, in order to avoid a double counting. 

More precisely to estimate the sample variance contribution due to
cosmic shear, we can populate exactly the same configurations of
lens-source galaxies into ray-tracing simulations, and


\begin{eqnarray}
\frac{\partial \ln P_{\dsigma}(k)}{\partial
 \delta_b}&=&-\frac{1}{\bar{N}_\bh(z_L)}\frac{\partial
 \bar{N}_\bh(z_L)}{\partial \delta_b}+
 \frac{1}{\bar{N}_\bh(z_L)}\int\!\dr\chi~\chi^2
 \frac{\partial }{\partial
 \delta_b}\left[\int\!\dr M~\frac{\dr n}{\dr M}S(M,\chi;z_L)P_{\rm
	   hm}(k;M)\right]\nonumber\\
 &=& -\bar{b}_1+\frac{1}{\bar{N}_\bh(z_L)}\int\!\dr\chi~\chi^2
 \frac{\partial }{\partial
 \delta_b}\left[\int\!\dr M~\frac{\dr n}{\dr M}S(M,\chi;z_L)P_{\rm
	   hm}(k;M)\right]\nonumber\\
\end{eqnarray}

\begin{equation}
 \bar{T}^{\rm 2D}_{{\rm hh}\dsigma\dsigma}(k,k')
  =\frac{\bar{\rho}_{\rm m0}^2}{\bar{N}_\bh(z_L)^2}
  \int\!\dr\chi~\chi^4\int\!\dr M\dr M'\frac{\dr n}{\dr M}\frac{\dr n}{\dr M'}S(M,\chi;z_L)S(M',\chi;z_L)
T_{{\rm hhmm}}(\bk,-\bk,\bk',-\bk')
\end{equation}

\begin{equation}
T=T^{\rm 1h}+\left(T^{\rm 2h(22)}+T^{\rm 2h(13)}\right)+T^{\rm
 3h}+T^{\rm 4h}
\end{equation}

\begin{eqnarray}
 &&T^{\rm 1h}_{\rm hmhm}(\bk,-\bk,\bk',-\bk')=\frac{\bar{\rho}_{\rm m0}^2}{\bar{N}_\bh(z_L)^2}
  \int\!\dr\chi~\chi^4
  \int\!\dr M~\frac{\dr n}{\dr M}S(M,\chi;z_L)^2\left(\frac{M}{\bar{\rho}_{\rm
					m0}}\right)^2
  u_M(k)u_M(k')\\
  &&T^{\rm 2h(13)}_{\rm hmhm}(\bk,-\bk,\bk',-\bk')=
  \frac{\bar{\rho}_{\rm m0}^2}{\bar{N}_\bh(z_L)^2}
  \int\!\dr\chi~\chi^4
  \int\!\dr M_1\dr M_2~\frac{\dr n}{\dr M_1}\frac{\dr n}{\dr M_2}\nonumber\\
&&\hspace{4em}\times  \left[
   P_{\rm hh}(k)\left\{S(M_1)S(M_2)\left(\frac{M_2}{\bar{\rho}_{\rm m0}}\right)^2
 u_{M_2}(k)u_{M_2}(k')
 +\frac{M_1}{\bar{\rho}_{\rm
 m0}}u_{M_1}(k)S(M_2)^2\frac{M_2}{\bar{\rho}_{\rm m0}}u_{M_2}(k')
		\right\} +(k\leftrightarrow k')  
  \right]\\
&&T^{\rm 2h(22)}_{\rm hmhm}(\bk,-\bk,\bk',-\bk')=
  \frac{\bar{\rho}_{\rm m0}^2}{\bar{N}_\bh(z_L)^2}
  \int\!\dr\chi~\chi^4
  \int\!\dr M_1\dr M_2~\frac{\dr n}{\dr M_1}\frac{\dr n}{\dr M_2}\nonumber\\
 &&\hspace{4em}\times
\left[
 P_{\rm hh}(|\bk+\bk'|)S(M_1)^2\left(\frac{M_2}{\bar{\rho}_{\rm
				m0}}\right)^2
 u_{M_2}(k)u_{M_2}(k')
 +P_{\rm hh}(|\bk-\bk'|)S(M_1)S(M_2)\frac{M_1}{\bar{\rho}_{\rm m0}}
 \frac{M_1}{\bar{\rho}_{\rm m0}}u_{M_1}(k')u_{M_2}(k)
	   \right]\\
&& T^{\rm 3h}_{\rm hmhm}(\bk,-\bk,\bk',-\bk')=
  \frac{\bar{\rho}_{\rm m0}^2}{\bar{N}_\bh(z_L)^2}
  \int\!\dr\chi~\chi^4
  \int\!\dr M_1\dr M_2\dr M_3~\frac{\dr n}{\dr M_1}\frac{\dr n}{\dr M_2}\frac{\dr n}{\dr M_3}\nonumber\\
 &&\hspace{4em}\times
  \left[
B_{\rm hhh}(\bk+\bk',-\bk,-\bk')S(M_1)^2\frac{M_2}{\bar{\rho}_{\rm
m0}}u_{M_2}(k)\frac{M_3}{\bar{\rho}_{\rm m0}}u_{M_3}(k')
  \right.
%\nonumber\\
% &&\hspace{6em}
  +B_{\rm hhh}(\bk-\bk',-\bk,\bk')S(M_1)\frac{M_1}{\bar{\rho}_{\rm
m0}}u_{M_1}(k')
\frac{M_2}{\bar{\rho}_{\rm m0}}u_{M_2}(k)S(M_3)
\nonumber\\
&&\hspace{4em}
 +B_{\rm hhh}(-\bk+\bk',\bk,-\bk')S(M_1)\frac{M_1}{\bar{\rho}_{\rm
m0}}u_{M_1}(k)S(M_2)
\frac{M_3}{\bar{\rho}_{\rm m0}}u_{M_3}(k')
%\nonumber\\
% &&\hspace{6em}
  \left.
		+B_{\rm hhh}(-\bk-\bk',\bk,\bk')
\left(\frac{M_1}{\bar{\rho}_{\rm m0}}\right)^2u_{M_1}(k)u_{M_1}(k')
S(M_2)S(M_3)\right]\nonumber\\
 &&\\
&& T^{\rm 4h}_{\rm hmhm}(\bk,-\bk,\bk',-\bk')=
  \frac{\bar{\rho}_{\rm m0}^2}{\bar{N}_\bh(z_L)^2}
  \int\!\dr\chi~\chi^4\nonumber\\
&&\hspace{6em}\times  \int\!\dr M_1\dr M_2\dr M_3\dr M_4~\frac{\dr n}{\dr M_1}\frac{\dr n}{\dr M_2}\frac{\dr n}{\dr M_3}
  \frac{\dr n}{\dr M_4}S(M_1)S(M_3)T_{\rm hhhh}(\bk,-\bk,\bk',-\bk')
\end{eqnarray}

\begin{eqnarray}
 &&T^{\rm 1h}_{\rm hmhm}(\bk,-\bk+\bq,\bk',-\bk'-\bq)\simeq
  T^{\rm 1h}_{\rm hmhm}(\bk,-\bk,\bk',-\bk')\\
 &&T^{\rm 2h(13)}_{\rm hmhm}(\bk,-\bk+\bq,\bk',-\bk'-\bq)
  \simeq T^{\rm 2h(22)}_{\rm hmhm}(\bk,-\bk,\bk',-\bk')
  +P_{\rm hh}(q)S(M_1)\frac{M_1}{\bar{\rho}_{\rm m0}}u_{M_1}(k)
  S(M_2)\frac{M_2}{\bar{\rho}_{\rm m0}}u_{M_2}(k')
\\
 &&T^{\rm 3h}_{\rm hmhm}(\bk,-\bk+\bq,\bk',-\bk'-\bq)
  \simeq T^{\rm 3h}(\bk,-\bk,\bk',-\bk')
  +B_{\rm hhh}(\bq,\bk',-\bk'-\bq)S(M_1)\frac{M_1}{\bar{\rho}_{\rm
  m0}}u_{M_1}(k)
  S(M_2)\frac{M_3}{\bar{\rho}_{\rm m0}}u_{M_3}(k')\nonumber\\
 &&\hspace{18em}+B_{\rm hhh}(-\bq,\bk,-\bk+\bq)
  S(M_1)\frac{M_1}{\bar{\rho}_{\rm
  m0}}u_{M_1}(k')S(M_2)\frac{M_3}{\bar{\rho}_{\rm m0}}u_{M_3}(k)\\
 &&T^{\rm 4h}_{\rm hmhm}(\bk,-\bk+\bq,\bk',-\bk'-\bq)
  \simeq T^{4h}_{\rm hhhh}(\bk,-\bk,\bk',-\bk')+.....
\end{eqnarray}

\begin{eqnarray}
B^{\rm PT}_{\rm m}(\bk_1,\bk_2,\bk_3)&=&2F_2(\bk_1,\bk_2)
  P^{\rm lin}_{\rm m}(k_1)P^{\rm lin}_{\rm m}(k_2)+2~{\rm perm.}\nonumber\\
T^{\rm PT}_{\rm m}(\bk_1,\bk_2,\bk_3,\bk_4)&=&
  4\left[F_2(\bk_{13},-\bk_1)F_2(\bk_{13},\bk_2)
    P^{\rm lin}_{\rm m}(k_{13})
    P^{\rm lin}_{\rm m}(k_{1})
    P^{\rm lin}_{\rm m}(k_{2})
 +11~{\rm perm.}\right]\nonumber \\
 &&+6\left[
      F_3(\bk_1,\bk_2,\bk_3)
          P^{\rm lin}_{\rm m}(k_{1})
          P^{\rm lin}_{\rm m}(k_{2})
          P^{\rm lin}_{\rm m}(k_{3})+3~{\rm perm.}
     \right]
\end{eqnarray}

\begin{align}
F_2(\bk_1,\bk_2) \equiv &\frac{5}{7}+\frac{1}{2}
\left(\frac{1}{k_1^2}+\frac{1}{k_2^2}\right)(\bk_1\cdot\bk_2)
+\frac{2}{7}\frac{(\bk_1\cdot\bk_2)^2}{k_1^2k_2^2},\nonumber\\
F_3(\bk_1,\bk_2,\bk_3) \equiv &
 \frac{7}{18}\frac{\bk_{12}\cdot\bk_1}{k_1^2}
 \left[F_2(\bk_2,\bk_3)+G_2(\bk_1,\bk_2)\right]
+\frac{1}{18}\frac{k_{12}^2(\bk_1\cdot\bk_2)}{k_1^2k_2^2}
\left[G_2(\bk_2,\bk_3)+G_2(\bk_1,\bk_2)\right], \nonumber\\
G_2(\bk_1,\bk_2)\equiv & \frac{3}{7}+\frac{1}{2}
\left(\frac{1}{k_1^2}+\frac{1}{k_2^2}\right)
(\bk_1\cdot\bk_2)+
\frac{4}{7}\frac{(\bk_1\cdot\bk_2)^2}{k_1^2k_2^2}
\end{align}


The lensing convergence field, more generally any two-dimensional field
projected along the line-of-sight, can be expressed as
%
\begin{equation}
\kappa(\btheta)=\int\!\dr\chi~W(\chi)\deltam(\chi,\chi\btheta)\simeq
 \sum_{i}\Delta\chi~ W(\chi_i)\deltam(\chi_i,\chi\btheta),
\end{equation}
%
where $W(\chi)$ is the radial weight function. To measaure a
cross-correlation of the projected field with the distribution of halos
at a particular redshfit $z_L$ \citep{Hikageetal:13,Nishizawaetal:13},
we can estimate the three-dimensional distribution of the underlying
mass density field at the halo redshift:
%
\begin{equation}
 \hat{\delta}_{\mathrm{m}}^\kappa(\bk_\perp;z_L)\equiv
  \int_S\! \dr^2\bx_\perp W^{-1}(z_L)\kappa(\btheta)e^{-i\bk_\perp\cdot
  \chi_L\btheta}
  \simeq \deltam(\bk_\perp,k_\parallel;\chi_L)+\chi_L^{-2}W^{-1}(z_L)\kappa_{\bl}
\end{equation}
%
where $\chi_L\equiv \chi(z_L)$ and $\bx_\perp$ and $\bk_\perp$ are the
two-dimensional spatial vector and wavevector in the plane perpendicular
to the line-of-sight direction.




\begin{eqnarray}
 \havedsigma(k)(2\pi)^2\delta_D^2(\bk+\bk')&\equiv &
  \ave{\hat{\delta}_\bh^{\rm 2D}(\bk,\chi)\sigmacr(\chi)}
\end{eqnarray}

The two-dimensional Fourier mode in the lens plane at redshift $z$ can
be estimted as
%
\begin{equation}
\hdsigma(\bk_\perp;\chi)\equiv\sigmacr(\chi)\int_{S=\chi\Omega_S}\!\dr^2\bx_\perp~\kappa(\btheta)e^{-i\bk_\perp\cdot\bx_\perp},
\end{equation}
%
where $\bx_\perp$ is the two-dimensional position vector in the 
plane at $z$, $\bx_\perp=\chi\btheta$.

\begin{equation}
 \hdsigma(\bk_\perp)=\chi_L^2 \kappa({\bl=\chi_L\bk_\perp})
\end{equation}
%
\begin{eqnarray}
 \ave{\hdsigma(\bk_\perp)\hdsigma(\bk'_\perp)}&\equiv&
  P_{\dsigma}(k_\perp)(2\pi)^2\delta_D^2(\bk_\perp+\bk'_\perp)\nonumber\\
&\leftarrow &\chi_L^4
 C_\kappa(l=\chi_Lk_\perp)(2\pi)^2\delta_D^2(\bl+\bl')|_{\bl=\chi_L\bk_\perp}\nonumber\\ 
&&=\chi_L^2 C_\kappa(l=\chi_Lk_\perp)(2\pi)^2\delta_D^2(\bk_\perp+\bk'_\perp)
\end{eqnarray}


Cross-correlating the convergence field on the sky with the distribution
of halos in the particular redshift slice around $z_L$ allows us to
probe the mass power spectrum at the redshift:
%
\begin{equation}
\dsigma(R;z_L)\equiv \ave{\delta_h^{\rm 2D}(\bx_\perp;z_L)\sigmacr(z_L)\kappa(\btheta)}_{R=|\bx_\perp-\chi_L\btheta|},
\end{equation}
%
where the average is doen over all the pairs of source galaxy and
lensing halo that are separate by the projected distance $R$ to within
the bin width, in the plane parpendicualr to the line-of-sight. The
stacked lensing profiles can be expressed in terms of teh power spectrum
as
%
\begin{eqnarray}
 \Sigma(R;z_L)&=&\int\!\frac{k\dr k}{2\pi}P_{\dsigma}(k;z_L)J_0(kR),\nonumber\\
 \dsigma(R;z_L)&=&\int\!\frac{k\dr k}{2\pi}P_{\dsigma}(k;z_L)J_2(kR),
\end{eqnarray}
%
where $J_0(x)$ and $J_2(x)$ are the zero-th and 2nd order Bessel
functions, and the power spectrum $P_{\dsigma}(k)$ is given as
%
\begin{equation}
 P_{\dsigma}(k)=\frac{\bar{\rho}_{\rm m0}}{\hat{N}_\bh(z_L)}
  \int\!\dr\chi~\chi^2
  \int\!\dr M~\frac{\dr n}{\dr M}S(M,z_L)P_{\rm hm}(k;M).
\end{equation}


Hence an estimtor of the matter power spectrum at $z_L$ can be
constructed from the cross-correlation function:
%
\begin{equation}
\hat{P}_{\mathrm{hm}}(k;z_L)=\ave{\deltah^\ast(\bk;z_L)\hat{\delta}^\kappa_{\mathrm{m}}(\bk)}
\end{equation}
%




\begin{equation}
 P_{\dsigma}(k;z_L)=
  \frac{1}{\bnh(z_L)}\bar{\rho}_{\rm m0}
  \int\!\dr M~\frac{\dr n}{\dr M}S(M;z_L)
\phm(k;M).
\end{equation}

\begin{equation}
 \dsigma(R;z_L)=\int\!\frac{k\dr k}{2\pi}P_{\dsigma}(k;z_L)J_2(kR).
\end{equation}

\begin{equation}
f(\bx_\perp)=\frac{1}{\chi_L^2\Omega_S}\sum_{\bk}
 f(\bk_\perp)e^{i\bk_\perp\cdot\bx_\perp}
 \simeq \int\!\frac{\dr^2\bk_\perp}{(2\pi)^2}f(\bk_\perp)e^{i\bk_\perp\cdot\bx_\perp}
\end{equation}

\begin{equation}
\hat{P}_{\rm hm}(k_i)=\frac{1}{\chi_L^2\Omega_S N_{\rm mode}(k)}\sum_{\bk;
 k\in k_i} \hat{\delta}_\bh(\bk)\hdsigma(-\bk)
\end{equation}

\begin{eqnarray}
{\rm Cov}[\phm(k_i),\phm(k_j)]^{\rm NG}&=&\frac{1}{\chi_L^4\Omega_S^2N_{\rm
 mode}(k)N_{\rm mode}(k')}\sum_{\bk; k\in k_i}
 \sum_{\bk'; k'\in k_j}
 \ave{\hat{\delta}_\bh(\bk)\hdsigma(-\bk)\hat{\delta}_\bh(\bk')\hdsigma(-\bk')}_{\rm
 c}\nonumber\\
&=&\frac{1}{\chi_L^4\Omega_S^2N_{\rm
 mode}(k)N_{\rm mode}(k')}\sum_{\bk; k\in k_i}
 \sum_{\bk'; k'\in k_j}(\chi_L^2\Omega_S)\delta^K_{\bk-\bk+\bk'-\bk'}
 T_{{\rm hh}\dsigma\dsigma}(\bk,-\bk,\bk',-\bk')\nonumber\\
 &=&\frac{1}{\chi_L^2\Omega_SN_{\rm
 mode}(k)N_{\rm mode}(k')}\sum_{\bk; k\in k_i}
 \sum_{\bk'; k'\in k_j}
 T_{{\rm hh}\dsigma\dsigma}(\bk,-\bk,\bk',-\bk')\nonumber\\
&=& \frac{\chi_L^2\Omega_S}{N_{\rm
 mode}(k)N_{\rm mode}(k')}
\int_{k\in k_i}\!\frac{\dr^2\bk}{(2\pi)^2}
\int_{k'\in k_j}\!\frac{\dr^2\bk'}{(2\pi)^2}
 T_{{\rm hh}\dsigma\dsigma}(\bk,-\bk,\bk',-\bk')\nonumber\\
&=& \frac{1}{\chi_L^2\Omega_S}
\int_{k\in k_i}\!\frac{\dr^2\bk}{2\pi k_i\Delta k}
\int_{k'\in k_j}\!\frac{\dr^2\bk'}{2\pi k_j\Delta k}
 T_{{\rm hh}\dsigma\dsigma}(\bk,-\bk,\bk',-\bk')\nonumber\\
&=& \frac{1}{4\pi f_{\rm sky}\chi_L^2}
\int_{k\in k_i}\!\frac{\dr^2\bk}{2\pi k_i\Delta k}
\int_{k'\in k_j}\!\frac{\dr^2\bk'}{2\pi k_j\Delta k}
 T_{{\rm hh}\dsigma\dsigma}(\bk,-\bk,\bk',-\bk')
\end{eqnarray}

\begin{eqnarray}
 \ave{\hat{\delta}^{\rm 2D}_\bh(\bk_1)
  \hat{\delta}^{\rm 2D}_\bh(\bk_2)
  \hdsigma(\bk_3)
  \hdsigma(\bk_4)
  }&=&\frac{\sigmacr(z_L)^2}{\bar{N}_\bh(z_L)^2}
  \int\!\dr\chi_1\dr\chi_2
  \dr\chi_3 \dr\chi_4
  \chi_1^2\chi_2^2 W_\kappa(\chi_3)W_\kappa(\chi_4)\nonumber\\
 &&\hspace{-8em}\times
  \int\!\dr M_1\dr M_2 \frac{\dr n}{\dr M_1}\frac{\dr n}{\dr M_2}
  S(M_1,\chi_1;z_L)S(M_2,\chi_2;z_L)\nonumber\\
 &&\hspace{-8em}\times
  \int\!\frac{\dr k_{1\parallel}}{2\pi}
  \cdots\frac{\dr k_{4\parallel}}{2\pi}
  \ave{\deltah(\bk_1)  \deltah(\bk_2)
  \delta_{\rm m}(\bk_3)  \delta_{\rm m}(\bk_4)}
  e^{ik_{1\parallel}\chi_1+\cdots+ik_{4\parallel}\chi_4
  }\nonumber\\
 &&\hspace{-14em}=(2\pi)^2\delta_D^2(\bk_{1\perp}+\cdot+\bk_{4\perp})
\frac{\sigmacr(z_L)^2}{\bar{N}_\bh(z_L)^2}
  \int\!\dr\chi_1\dr\chi_2
  \dr\chi_3 \dr\chi_4
  \chi_1^2\chi_2^2 W_\kappa(\chi_3)W_\kappa(\chi_4)\nonumber\\ 
 &&\hspace{-8em}\times
  \int\!\dr M_1\dr M_2 \frac{\dr n}{\dr M_1}\frac{\dr n}{\dr M_2}
  S(M_1,\chi_1;z_L)S(M_2,\chi_2;z_L)\nonumber\\
&&\hspace{-8em}\times \int\!\frac{\dr k_{1\parallel}}{2\pi}
  \cdots\frac{\dr k_{3\parallel}}{2\pi}
 T_{\rm hhmm}(\bk_1,\bk_2,\bk_3,\bk_4)
  e^{ik_{1\parallel}(\chi_1-\chi_4)+i\bk_{2\parallel}(\chi_2-\chi_4)
  +i\bk_{3\parallel}(\chi_3-\chi_4)
  }\nonumber\\
 &&\hspace{-14em}\simeq (2\pi)^2\delta_D^2(\bk_{1\perp}+\cdot+\bk_{4\perp})
\frac{\sigmacr(z_L)^2}{\bar{N}_\bh(z_L)^2}
  \int\!\dr\chi
  \chi^4 W_\kappa(\chi)^2
  \int\!\dr M_1\dr M_2 \frac{\dr n}{\dr M_1}\frac{\dr n}{\dr M_2}
  S(M_1,\chi_1;z_L)S(M_2,\chi_2;z_L)
 T_{\rm hhmm}(\bk_1,\bk_2,\bk_3,\bk_4;M_1,M_2,\chi_L)
 \nonumber\\
 &&\hspace{-14em}\simeq
  (2\pi)^2\delta_D^2(\bk_{1\perp}+\cdot+\bk_{4\perp})
  \frac{\bar{\rho}_{\rm m0}^2}{\bar{N}_\bh(z_L)^2}
 \chi_L^4 \Delta\chi  \int\!\dr M_1\dr M_2 \frac{\dr n}{\dr M_1}\frac{\dr n}{\dr M_2}
  S(M_1,\chi_1;z_L)S(M_2,\chi_2;z_L)
 T_{\rm hhmm}(\bk_1,\bk_2,\bk_3,\bk_4;\chi_L)\nonumber\\
 &&\hspace{-14em}=
  (2\pi)^2\delta_D^2(\bk_{1\perp}+\cdot+\bk_{4\perp})
  \frac{\bar{\rho}_{\rm m0}^2}{\bnh(z_L)^2\Delta\chi}
\int\!\dr M_1\dr M_2 \frac{\dr n}{\dr M}\frac{\dr n}{\dr M}
  S(M,z_L)S(M',z_L)
 T_{\rm hhmm}(\bk_1,\bk_2,\bk_3,\bk_4;M,M',\chi_L)
\end{eqnarray}

\begin{eqnarray}
 &&\delta_\bh(\bx_1)=\frac{1}{\bar{n}_\bh}n_\bh(\bx_1)-1=\frac{1}{\bnh}\sum_{i}\delta_D^3(\bx_i-\bx_1)-1\nonumber\\
 &&\deltam(\bx_1)=\frac{1}{\bar{\rho}_{\rm m0}}\sum_i m_iu_{m_i}(\bx_1-\bx_i)-1
\end{eqnarray}



\begin{eqnarray}
 \xi_{\rm 4, hhmm}(\bx_1,\bx_2,\bx_3,\bx_4)&=&
  \ave{\delta_\bh(\bx_1)\delta_\bh(\bx_2)\delta_{\rm
  m}(\bx_3)\delta_{\rm m}(\bx_4)}
\end{eqnarray}

\begin{eqnarray}
\bnh^2\rho_{\rm m0}^2 \xi^{{\rm 1h}}_{\rm 4,
 hhmm}(\bx_1,\bx_2,\bx_3,\bx_4)&=&
 \ave{\sum_i \delta_D^3(\bx_1-\bx_i)\delta_D^3(\bx_2-\bx_i)
 M_iu_{M_i}(\bx_3-\bx_i)M_iu_{M_i}(\bx_4-\bx_i)
 }\nonumber\\
 &&\hspace{-8em}=\int\!\dr M \frac{\dr n}{\dr M}M^2 \int\!d\bx \delta_D^3(\bx_1-\bx)\delta_D^3(\bx_2-\bx)
 u_{M}(\bx_3-\bx)u_{M}(\bx_4-\bx)\nonumber\\
 &&\hspace{-8em}=\int\!\dr M \frac{\dr n}{\dr M}M^2 \int\!d\bx
  \int\!\frac{d^3\bk_1}{(2\pi)^3}\cdots\frac{d^3\bk_4}{(2\pi)^3}
  e^{i(\bk_1\cdot\bx_1+\cdots+\bk_4\cdot\bx_4)}e^{-i\bx\cdot(\bk_1+\cdot+\bk_4)}
 u_{M}(k_3)u_{M}(k_4)\nonumber\\
 &&\hspace{-8em}=\int\!\dr M \frac{\dr n}{\dr M}M^2 
  \int\!\frac{d^3\bk_1}{(2\pi)^3}\cdots\frac{d^3\bk_4}{(2\pi)^3}
  e^{i(\bk_1\cdot\bx_1+\cdots+\bk_4\cdot\bx_4)}
  u_{M}(k_3)u_{M}(k_4)(2\pi)^3\delta_D^3(\bk_1+\cdots+\bk_4)
  \nonumber\\
 &&\hspace{-8em}\rightarrow T^{\rm 1h}_{\rm hhmm}(\bk_1,\bk_2,\bk_3,\bk_4)=\frac{1}{\bnh^2}
  \int\!\dr M\frac{\dr n}{\dr M}\left(\frac{M}{\bar{\rho}_{\rm m0}}\right)^2 u_M(k_3)u_M(k_4)
\end{eqnarray}


\begin{eqnarray}
\bnh^2\rho_{\rm m0}^2 \xi^{{\rm 2h-(13)}}_{\rm 4,
 hhmm}(\bx_1,\bx_2,\bx_3,\bx_4)&=&
 \ave{\sum_{i,j;i\ne j} \delta_D^3(\bx_1-\bx_i)\delta_D^3(\bx_2-\bx_j)
 M_ju_{M_j}(\bx_3-\bx_j)M_ju_{M_j}(\bx_4-\bx_j)
 }\nonumber\\
&& +\ave{\sum_{i,j; i\ne j}M_i
 u_{M_i}(\bx_3-\bx_i)\delta_D^3(\bx_1-\bx_j)\delta_D^3(\bx_2-\bx_j)
 M_j u_{M_j}(\bx_4-\bx_j)}+\mbox{(perm.)}\nonumber\\
 &&\hspace{-12em}=\int\!d^3\by_1
  d^3\by_2 \int\!\dr M_1 \dr M_2\frac{\dr n}{\dr M_1}\frac{\dr n}{\dr M_2}
 \xi_{\rm
hh}(\by_1-\by_2;M_1,M_2) \left[\delta_D^3(\bx_1-\by_1)\delta_D^3(\bx_2-\by_2)
M_2^2u_{M_2}(\bx_3-\by_2)u_{M_2}(\bx_4-\by_2)
  \right.\nonumber\\
 &&\hspace{-8em}\left.
+M_1u_{M_1}(\bx_3-\by_1)\delta_D^3(\bx_1-\by_2)\delta_D^3(\bx_2-\by_2)
M_2u_{M_2}(\bx_4-\by_2)+(\mbox{perm.})
 \right] \nonumber\\
 &&\hspace{-12em}=\int\!d^3\by_1
 d^3\by_2 \int\!\dr M_1 \dr M_2\frac{\dr n}{\dr M_1}\frac{\dr n}{\dr M_2}
 \int\!\frac{d^3\bq}{(2\pi)^3}
 P_{hh}(q;M_1,M_2)e^{i\bq\cdot(\by_1-\by_2)}\nonumber\\
&&\hspace{-10em} \times\int\!\frac{d^3\bk_1}{(2\pi)^3}\cdots\frac{d^3\bk_4}{(2\pi)^3}
e^{i(\bk_1\cdot\bx_1+\cdots+\bk_4\cdot\bx_4)}
\left[e^{-i\bk_1\cdot\by_1}e^{-i\by_2\cdot(\bk_2+\bx_3+\bk_4)}
M_2^2u_{M_2}(k_3)u_{M_2}(k_4)
  \right.\nonumber\\
 &&\hspace{-8em}\left.
		 +e^{-i\by_1\cdot\bk_3}e^{-i\by_2\cdot(\bk_1+\bk_2+\bk_4)}M_1u_{M_1}(k_3)
		 M_2u_{M_2}(k_4)+(\mbox{perm.})
 \right] \nonumber\\
 &&\hspace{-12em}=
 \int\!\dr M_1 \dr M_2\frac{\dr n}{\dr M_1}\frac{\dr n}{\dr M_2}
 \int\!\frac{d^3\bq}{(2\pi)^3}
 P_{hh}(q;M_1,M_2)
 \nonumber\\
&&\hspace{-10em} \times\int\!\frac{d^3\bk_1}{(2\pi)^3}\cdots\frac{d^3\bk_4}{(2\pi)^3}
e^{i(\bk_1\cdot\bx_1+\cdots+\bk_4\cdot\bx_4)}
\left[(2\pi)^3\delta_D^3(\bq-\bk_1)(2\pi)^3\delta_D^3(\bq+\bk_2+\bk_3+\bk_4)
M_2^2u_{M_2}(k_3)u_{M_2}(k_4)
  \right.\nonumber\\
 &&\hspace{-8em}\left.
		 +(2\pi)^3\delta_D^3(\bq-\bk_3)(2\pi)^3\delta_D^3(\bq+\bk_1+\bk_2+\bk_4)
		 		 M_1u_{M_1}(k_3)
		 M_2u_{M_2}(k_4)+(\mbox{perm.})
 \right] \nonumber\\
 &&\hspace{-12em}=
  \int\!\dr M_1 \dr M_2\frac{\dr n}{\dr M_1}\frac{\dr n}{\dr M_2}
  \int\!\frac{d^3\bk_1}{(2\pi)^3}\cdots\frac{d^3\bk_4}{(2\pi)^3}
(2\pi)^3\delta_D^3(\bk_{1234})e^{i(\bk_1\cdot\bx_1+\cdots+\bk_4\cdot\bx_4)}
 \nonumber\\
&&\hspace{-10em} \times
 \left[P_{\rm hh}(k_1)M_2^2u_{M_2}(k_3)u_{M_2}(k_4)
  + P_{\rm hh}(k_3)M_1M_2u_{M_1}(k_3)u_{M_2}(k_4)
+(\mbox{perm.})
		    \right] \nonumber\\
 &&\hspace{-12em}\rightarrow
  T^{\rm 2h(13)}_{\rm 4, hhmm}(\bk_1,\bk_2,\bk_3,\bk_4)=
  \int\!\dr M_1 \dr M_2\frac{\dr n}{\dr M_1}\frac{\dr n}{\dr M_2}
  \left[\left\{P_{\rm hh}(k_1)+P_{\rm hh}(k_2)\right\}
   M_2^2u_{M_2}(k_3)u_{M_2}(k_4)\right.\nonumber\\
&&\hspace{0em} \left. +P_{\rm hh}(k_3)M_1u_{M_1}(k_3)M_2u_{M_2}(k_4)
   +P_{\rm hh}(k_4)M_1u_{M_1}(k_4)M_2u_{M_2}(k_4)
  \right]
\end{eqnarray}

\begin{eqnarray}
\bnh^2\rho_{\rm m0}^2 \xi^{{\rm 2h-(22)}}_{\rm 4,
 hhmm}(\bx_1,\bx_2,\bx_3,\bx_4)&=&
 \ave{\sum_{i,j;i\ne j} \delta_D^3(\bx_1-\bx_i)\delta_D^3(\bx_2-\bx_i)
 M_ju_{M_j}(\bx_3-\bx_j)M_ju_{M_j}(\bx_4-\bx_j)
 }\nonumber\\
 && +\ave{\sum_{i,j; i\ne j}
\delta_D^3(\bx_1-\bx_i)\delta_D^3(\bx_2-\bx_i)
M_i
 u_{M_j}(\bx_3-\bx_j)
 M_j u_{M_j}(\bx_4-\bx_j)}+\mbox{(perm.)}\nonumber\\
 &&\hspace{-12em}=\int\!d^3\by_1
  d^3\by_2 \int\!\dr M_1 \dr M_2\frac{\dr n}{\dr M_1}\frac{\dr n}{\dr M_2}
 \xi_{\rm
hh}(\by_1-\by_2;M_1,M_2) \left[\delta_D^3(\bx_1-\by_1)\delta_D^3(\bx_2-\by_1)
M_2^2u_{M_2}(\bx_3-\by_2)u_{M_2}(\bx_4-\by_2)
  \right.\nonumber\\
 &&\hspace{-8em}\left.
		 +\delta_D^3(\bx_1-\by_1)M_1u_{M_1}(\bx_3-\by_1)
		 \delta_D^3(\bx_2-\by_2)M_2u_{M_2}(\bx_4-\by_2)
+(\mbox{perm.})
 \right] \nonumber\\
 &&\hspace{-12em}=\int\!d^3\by_1
 d^3\by_2 \int\!\dr M_1 \dr M_2\frac{\dr n}{\dr M_1}\frac{\dr n}{\dr M_2}
 \int\!\frac{d^3\bq}{(2\pi)^3}
 P_{hh}(q;M_1,M_2)e^{i\bq\cdot(\by_1-\by_2)}\nonumber\\
&&\hspace{-10em} \times\int\!\frac{d^3\bk_1}{(2\pi)^3}\cdots\frac{d^3\bk_4}{(2\pi)^3}
e^{i(\bk_1\cdot\bx_1+\cdots+\bk_4\cdot\bx_4)}
\left[e^{-i\by_1\cdot(\bk_1+\bk_2)}
  e^{-i\by_2\cdot(\bk_3+\bk_4)}
M_2^2u_{M_2}(k_3)u_{M_2}(k_4)
  \right.\nonumber\\
 &&\hspace{-8em}\left.
		 +e^{-i\by_1\cdot(\bk_1+\bk_3)}e^{-i\by_2\cdot(\bk_2+\bk_4)}M_1u_{M_1}(k_3)
		 M_2u_{M_2}(k_4)+(\mbox{perm.})
 \right] \nonumber\\
 &&\hspace{-12em}=
 \int\!\dr M_1 \dr M_2\frac{\dr n}{\dr M_1}\frac{\dr n}{\dr M_2}
 \int\!\frac{d^3\bq}{(2\pi)^3}
 P_{hh}(q;M_1,M_2)
 \nonumber\\
&&\hspace{-10em} \times\int\!\frac{d^3\bk_1}{(2\pi)^3}\cdots\frac{d^3\bk_4}{(2\pi)^3}
e^{i(\bk_1\cdot\bx_1+\cdots+\bk_4\cdot\bx_4)}
\left[(2\pi)^3\delta_D^3(\bq-\bk_1-\bk_2)(2\pi)^3\delta_D^3(\bq+\bk_3+\bk_4)
M_2^2u_{M_2}(k_3)u_{M_2}(k_4)
  \right.\nonumber\\
 &&\hspace{-8em}\left.
		 +(2\pi)^3\delta_D^3(\bq-\bk_1-\bk_3)(2\pi)^3\delta_D^3(\bq+\bk_2+\bk_4)
		 		 M_1u_{M_1}(k_3)
		 M_2u_{M_2}(k_4)+(\mbox{perm.})
 \right] \nonumber\\
 &&\hspace{-12em}=
  \int\!\dr M_1 \dr M_2\frac{\dr n}{\dr M_1}\frac{\dr n}{\dr M_2}
  \int\!\frac{d^3\bk_1}{(2\pi)^3}\cdots\frac{d^3\bk_4}{(2\pi)^3}
(2\pi)^3\delta_D^3(\bk_{1234})e^{i(\bk_1\cdot\bx_1+\cdots+\bk_4\cdot\bx_4)}
 \nonumber\\
&&\hspace{-10em} \times
 \left[P_{\rm hh}(k_{12})M_2^2u_{M_2}(k_3)u_{M_2}(k_4)
  + P_{\rm hh}(k_{13})M_1M_2u_{M_1}(k_3)u_{M_2}(k_4)
+(\mbox{perm.})
		    \right] \nonumber\\
 &&\hspace{-12em}\rightarrow
  T^{\rm 2h(22)}_{\rm 4, hhmm}(\bk_1,\bk_2,\bk_3,\bk_4)=
  \int\!\dr M_1 \dr M_2\frac{\dr n}{\dr M_1}\frac{\dr n}{\dr M_2}
  \left[P_{\rm hh}(k_{12})M_2^2u_{M_2}(k_3)u_{M_2}(k_4)\right.\nonumber\\
&&\left.   +\left\{P_{\rm hh}(k_{13})+P_{\rm hh}(k_{14})\right\}
   M_1u_{M_1}(k_3)M_2u_{M_2}(k_4)
  \right]
\end{eqnarray}

\begin{eqnarray}
\bnh^2\rho_{\rm m0}^2 \xi^{{\rm 3h}}_{\rm 4,
 hhmm}(\bx_1,\bx_2,\bx_3,\bx_4)&=&
 \ave{\sum_{i,j,k} \delta_D^3(\bx_1-\bx_i)\delta_D^3(\bx_2-\bx_i)
 M_ju_{M_j}(\bx_3-\bx_j)M_ku_{M_k}(\bx_4-\bx_k)
 }\nonumber\\
 && +\ave{\sum_{i,j,k}
\delta_D^3(\bx_1-\bx_i)
M_i
u_{M_i}(\bx_3-\bx_i)
\delta_D^3(\bx_2-\bx_j)
 M_k u_{M_k}(\bx_4-\bx_k)}+\mbox{(perm.)}\nonumber\\
 &&\hspace{-12em}=\int\!d^3\by_1
  d^3\by_2d^3\by_3 \int\!\dr M_1 \dr M_2\dr M_3\frac{\dr n}{\dr M_1}\frac{\dr n}{\dr M_2}\frac{\dr n}{\dr M_3}
  \int\!\frac{d^3\bq_1}{(2\pi)^3}\frac{d^3\bq_2}{(2\pi)^3}
  B_{\rm hhh}(\bq_1,\bq_2,-\bq_{12})e^{i\bq_1\cdot(\by_1-\by_3)+i\bq_2\cdot(\by_2-\by_3)}\nonumber\\
 &&\hspace{-10em}
 \times \left[\delta_D^3(\bx_1-\by_1)\delta_D^3(\bx_2-\by_1)
M_2u_{M_2}(\bx_3-\by_2)M_3u_{M_3}(\bx_4-\by_3)
  \right.\nonumber\\
 &&\hspace{-8em}
		 +\delta_D^3(\bx_1-\by_1)M_1u_{M_1}(\bx_3-\by_1)
		 \delta_D^3(\bx_2-\by_2)M_2u_{M_2}(\bx_4-\by_3)\nonumber\\
 &&\hspace{-8em}\left.
+M_1^2u_{M_1}(\bx_3-\by_1)u_{M_1}(\bx_4-\by_1)\delta_D^3(\bx_1-\by_2)\delta_D^3(\bx_1-\by_3)
 \right] \nonumber\\
 &&\hspace{-12em}=\int\!d^3\by_1
  d^3\by_2d^3\by_3
  \int\!\dr M_1 \dr M_2\dr M_3\frac{\dr n}{\dr M_1}\frac{\dr n}{\dr M_2}\frac{\dr n}{\dr M_3}
  \int\!\frac{d^3\bq_1}{(2\pi)^3}\frac{d^3\bq_2}{(2\pi)^3}
    B_{\rm hhh}(\bq_1,\bq_2,-\bq_{12})e^{i\bq_1\cdot(\by_1-\by_3)+i\bq_2\cdot(\by_2-\by_3)}\nonumber\\
&&\hspace{-10em} \times\int\!\frac{d^3\bk_1}{(2\pi)^3}\cdots\frac{d^3\bk_4}{(2\pi)^3}
e^{i(\bk_1\cdot\bx_1+\cdots+\bk_4\cdot\bx_4)}
\left[e^{-i\by_1\cdot(\bk_1+\bk_2)}
  e^{-i\by_2\cdot\bk_3}  e^{-i\by_3\cdot\bk_4}
M_2u_{M_2}(k_3)M_3u_{M_3}(k_4)
  \right.\nonumber\\
 &&\hspace{-8em}
  +e^{-i\by_1\cdot(\bk_1+\bk_3)}e^{-i\by_2\cdot\bk_2}
  e^{-i\by_3\cdot\bk_4}
  M_1u_{M_1}(k_3)
		 M_3u_{M_3}(k_4)\nonumber\\
 &&\hspace{-8em}
  \left. +e^{-i\by_1\cdot(\bk_3+\bk_4)}
        e^{-i\by_2\cdot\bk_1}
        e^{-i\by_3\cdot\bk_2}
   M_1^2u_{M_1}(k_3)u_{M_1}(k_4)
 \right] \nonumber\\
 &&\hspace{-12em}=
  \int\!\dr M_1 \dr M_2\dr M_3\frac{\dr n}{\dr M_1}\frac{\dr n}{\dr M_2}\frac{\dr n}{\dr M_3}
  \int\!\frac{d^3\bq_1}{(2\pi)^3}\frac{d^3\bq_2}{(2\pi)^3}
    B_{\rm hhh}(\bq_1,\bq_2,-\bq_{12})\nonumber\\
&&\hspace{-10em} \times\int\!\frac{d^3\bk_1}{(2\pi)^3}\cdots\frac{d^3\bk_4}{(2\pi)^3}
e^{i(\bk_1\cdot\bx_1+\cdots+\bk_4\cdot\bx_4)}
\left[(2\pi)^9\delta_D^3(\bq_1-\bk_1-\bk_2)\delta_D^3(\bq_2-\bk_3)\delta_D^3(\bq_1+\bq_2+\bk_4)
M_2u_{M_2}(k_3)M_3u_{M_3}(k_4)
  \right.\nonumber\\
 &&\hspace{-8em}
  +(2\pi)^9\delta_D^3(\bq_1-\bk_1-\bk_3)
  \delta_D^3(\bq_2-\bk_2)
  \delta_D^3(\bq_1+\bq_2+\bk_4)
  M_1u_{M_1}(k_3)
		 M_3u_{M_3}(k_4)\nonumber\\
 &&\hspace{-8em}
  \left. +
   (2\pi)^9\delta_D^3(\bq_1-\bk_3-\bk_4)\delta_D^3(\bq_2-\bk_1)
   \delta_D^3(\bq_1+\bq_2+\bk_2)
   M_1^2u_{M_1}(k_3)u_{M_1}(k_4)
 \right] \nonumber\\
 &&\hspace{-12em}=
  \int\!\dr M_1 \dr M_2\dr M_3\frac{\dr n}{\dr M_1}\frac{\dr n}{\dr M_2}\frac{\dr n}{\dr M_3}
  \int\!\frac{d^3\bk_1}{(2\pi)^3}\cdots \frac{d^3\bk_4}{(2\pi)^3}
  (2\pi)^3\delta_D^3(\bk_{1234})e^{i(\bk_1\cdot\bx_1+\cdots+\bk_4\cdot\bx_4)}
    \nonumber\\
&&\hspace{-10em} \times
\left[B_{\rm hhh}(\bk_{12},\bk_3,\bk_4)
M_2u_{M_2}(k_3)M_3u_{M_3}(k_4)
  +B_{\rm hhh}(\bk_{13},\bk_2,\bk_4)
  M_1u_{M_1}(k_3)
		 M_3u_{M_3}(k_4)\right.\nonumber\\
 &&\hspace{-8em}
  \left. +B_{\rm hhh}(\bk_{34},\bk_1,\bk_2)
   M_1^2u_{M_1}(k_3)u_{M_1}(k_4)
 \right] \nonumber\\
 &&\hspace{-14em}\rightarrow
  T^{3h}_{\rm hhh}(\bk_1,\bk_2,\bk_3,\bk_4)
  =  \int\!\dr M_1 \dr M_2\dr M_3\frac{\dr n}{\dr M_1}\frac{\dr n}{\dr M_2}\frac{\dr n}{\dr M_3}
\left[B_{\rm hhh}(\bk_{12},\bk_3,\bk_4)
M_2u_{M_2}(k_3)M_3u_{M_3}(k_4)\right.\nonumber\\
&&\hspace{-3em}\left.  +B_{\rm hhh}(\bk_{13},\bk_2,\bk_4)
  M_1u_{M_1}(k_3)
		 M_3u_{M_3}(k_4)
  +B_{\rm hhh}(\bk_{34},\bk_1,\bk_2)
   M_1^2u_{M_1}(k_3)u_{M_1}(k_4)+(\mbox{perm.})
 \right] \nonumber\\
\end{eqnarray}



\begin{equation}
 \Omega_S=\int\!\left[
		 \prod_i^n\frac{d\bq_i}{(2\pi)^2}
   \right]\tW(\bq_1)tW(\bq_2)\cdots\tW(\bq_n)
 =\int\!\left[
	 \prod_i^{(n-1)}\frac{d\bq_i}{(2\pi)^2}\right]
	 \tW(\bq_1)\tW(\bq_2)\cdots\tW(-\bq_1-\bq_2-
	 \cdots-\bq_{n-1})
\end{equation}



%
\begin{eqnarray}
  \hat{N}_{\bh}(z_L)&\equiv& \int\!\dr^2\btheta~W(\btheta)\int\!\dr\chi
  ~\frac{\dr^2V}{\dr\chi d\Omega} \int\!\dr M~ S(M)f_\bh(\chi;z_L)
  n_\bh(\bx;M)\nonumber\\
  &=&\int\!\dr^2\btheta~W(\btheta)\int\!\dr\chi~\chi^2
  f_\bh(\chi;z_L)\int\!\dr M~ S(M)\bnh(M;\chi)\left[
  1+b(M)\deltam(\chi,\chi\btheta)\right]\nonumber\\ &=&\Omega_S
  \int\!\dr\chi~\chi^2 f_\bh(\chi;z_L)\int\!\dr M~ S(M)\bnh(M;\chi)\left[
  1+b(M) \deltab(\chi)\right], \label{eq:eN}
\end{eqnarray}
%
where quantities with hat notation $\hat{\hspace{1em}}$, here and
hereafter, denote their estimator, the quantity with bar notation
denotes the ensemble average expectation, and
the background mode is
defined as
%
\begin{equation}
\deltab(\chi)\equiv
 \frac{1}{\Omega_S}\int\!\dr^2\btheta~W(\btheta)\deltam(\chi,\chi\btheta)
 =\frac{1}{A_S(\chi)}\int\!\dr^2\bx_\perp ~W(\bx_\perp)\deltam(\chi,\bx_\perp),
%\label{eq:deltab}
\end{equation}
%
$\bx_\perp\equiv \chi\btheta$, and $A_S(\chi)\equiv \chi^2\Omega_S$ is
the effective survey area of the plane at distance $\chi$, in units of
[${\rm Mpc}^2$]. Eq.~(\ref{eq:deltab}) shows that the background mode at
each redshift is from the average of Fourier modes perpendicular to the
line-of-sight direction. If ignoring the effect of radial Fourier mode
on clustering observables, whcih is a good approximation for lensing
observables, the background mode can be realized to cause a shift in the
mean mass density in each redshift plane. Note that, throughout this
paper, we employ a flat-geometry universe, where the radial and
angular-diameter distances are the same, and we emloy the flat sky
approximation.  The esemble average expectation can be compuated as
%
\begin{eqnarray}
 \bar{N}_\bh(z_L)&\equiv& \ave{\hat{N}_{\bh}(z_L)}=
\int\!\dr^2\btheta~W(\btheta)\int\!\dr\chi~\chi^2 f_\bh(\chi;z_L)\int\!\dr M~
  S(M)\bnh(M;\chi)\left[
		  1+b(M)\ave{\deltam(\chi,\chi\btheta)}\right]
  \nonumber\\
 &=&\Omega_S\int\!\dr\chi~\chi^2 f_\bh(\chi;z_L)\int\!\dr M~
  S(M)\bnh(M;\chi),
\end{eqnarray}
%
where we have uesd the fact $\ave{\deltab}=0$.

Similarly, taking into account the survey window, we define an estimator
of the number desnity fluctuation field of halos of mass $M$ and at
redschift $\chi$ as
%
\begin{equation}
\dnh^W(\bx_\perp; M,
 \chi)=\bnh(M;\chi)b(M)W(\btheta)\deltam(\chi,\chi\btheta).
 \label{eq:dnh}
\end{equation}
%
Since we later consider the projected field, we focus on the field in
the two-diemnsinal plane perpendicular to the line-of-sight direction;
hence, a mapping between the angular position and the comoving,
perpendicular position vector for each halo is given as
$\bx_\perp=\chi\btheta$. This is feasible as redshift of each is assumed
to be known. Integrating the above field in the light cone, weighted by
the selection function yields an estimator of the projected number
density fluctuation field:
%
\begin{equation}
 \sigmah^W(\btheta)\equiv \int\!\dr\chi~\chi^2 f_\bh(\chi;z_L)\int\!\dr M~
  S(M)\dnh^W(M;\chi).
  \end{equation}
 %
 The Fourier transform of $\dnh$ is
 %
\begin{equation}
 \tdnh^W(\bk_\perp,\chi)\equiv \int\!\dr^2\bx_\perp~\dnh^W(\bx_\perp)e^{-i\bx_\perp\cdot\bk_\perp}
  =\bnh(M;\chi)
  b(M)\int\!\frac{\dr k_\parallel}{2\pi}\tdeltam^W(\bk_\perp,k_\parallel;\chi)
  e^{i\chi k_\parallel},
\end{equation}
%
where
%
\begin{equation}
 \tdeltam^W(\bk,k_\parallel;\chi)\equiv 
  \int\!\frac{\dr^2\bq}{(2\pi)^2}\tW(\bl=\chi\bq)
  \tdeltam(k_\parallel,\bk_\perp-\bq),
\end{equation}
%
and $\tW(\bq;\chi)\equiv
\int\!\dr^2\bx_\perp~W(\bx)e^{-i\bq\cdot\bx_\perp}$ with $\bx_\perp=\chi\btheta$.



\begin{equation}
\hphh(k_{\perp,i})\equiv
 \frac{1}{\hat{N}_\bh(z_L)^2}\int\!\dr\chi\int\!\dr\chi'
f_\bh(\chi)f_\bh(\chi')\frac{1}{\chi^2\Omega_S}\int\!\frac{\dr^2\bk_{\perp}}{A(k_{\perp
i})} \tdnh^W(\bk_\perp,\chi)\tdnh^W(-\bk_\perp,\chi')
\end{equation}


Let us define an estimator of the projected power spectrum of halos in
the light cone as
%
\begin{equation}
 \hphh(k_{\perp, i})=\frac{1}{\Omega_S\hat{N}_\bh(z_L)^2}
  \int\!\dr\chi\int\!\dr\chi' \chi^2\chi^{\prime 2}f_\bh(\chi;z_L)f_\bh(\chi';z_L)
  \int\!\dr M\int\!\dr M'~S(M)S(M')\int_{k_{\perp
  i}}\!\frac{\dr^2\bk_\perp}{V(k_{\perp, i})} \tdnh^W(\bk_\perp,\chi) \tdnh^W(-\bk_\perp,\chi)
\end{equation}
%

\begin{eqnarray}
 \ave{ \hphh(k_{\perp, i})}&=&
  \frac{1}{\Omega_S\hat{N}_\bh(z_L)^2}
  \int\!\dr\chi\int\!\dr\chi' \chi^2\chi^{\prime 2}f_\bh(\chi;z_L)f_\bh(\chi';z_L)
  \int\!\dr M\int\!\dr M'~S(M)S(M')\chi^2\chi^{\prime
  2}b(M)\bnh(M) b(M')
  \bnh(M')
\nonumber\\
&&\times  
  \int\!\frac{\dr^2\bq}{(2\pi)^2}
  \int\!\frac{\dr^2\bq'}{(2\pi)^2}
    \int_{k_{\perp
 i}}\!\frac{\dr^2\bk_\perp}{V(k_{\perp, i})}
\int\!\frac{\dr k_\parallel}{2\pi}
    \tW_{\bq}\tW_{\bq'}
    P(|\bk-\bq|,k_\parallel)e^{ik_\parallel(\chi-\chi')}(2\pi)^2\delta_D^2(\bq-\bq')\nonumber\\
 &\simeq &
      \frac{1}{\Omega_S\hat{N}_\bh(z_L)^2}
  \int\!\dr\chi \chi^4f_\bh(\chi;z_L)^2
  \left[\int\!\dr M~S(M)b(M)\bnh(M)\right]^2\chi^4P(k_{\perp,
  i})\int\!\frac{\dr^2\bq}{(2\pi)^2}|\tW(\bl=\chi\bq)|^2
\end{eqnarray}


In the following, for simplicity we consider the uniform radial
selection given by
%
\begin{equation}
 f_\bh(\chi;z_L)=\Theta(\chi_L-\chi+\Delta\chi/2)\Theta(\chi_L+\chi-\Delta\chi/2),
\end{equation}
%
where $\Theta(x)$ is the Heviside-step function, defined as
$\Theta(x)=1$ if $x>0$ otherwise $\Theta(x)=0$, $\chi_L$ is the radial
comoving distance to $z_L$, $\chi_L\equiv \chi(z_L)$, and $\Delta\chi$
is the radial bin width.


For a reasonably narrow radial bin of the $z_L$-redshift slice,
satisfying $\Delta\chi/\chi_L\ll 1$, the cumulative number counts can be
simplified as
%
\begin{equation}
 \hat{N}_\bh(z_L)\simeq \bar{N}_{\bh}(z_L)\left[1+\bar{b}_1\delta_\br\right]
\end{equation}

\begin{equation}
 \delta^W_{\bh}(\bx_\perp;\chi)\equiv W(\chi\btheta)\delta^{\rm 2D}_{\bh}(\bx_\perp,\chi)
\end{equation}

\begin{equation}
\tdelta^W_{\bh}(\bk_\perp)=\int\!\frac{\dr^2\bq}{(2\pi)^2}\tW(\bq)\tdelta_{\bh}(\bk_\perp-\bq)
\end{equation}

In this paper we consider a catalog of halos, probed with galaxies
and/galaxy clusters. The average, projected number density of halos is
given in terms of the halo mass function as
%
\begin{eqnarray}
\bar{N}_{\mathrm{h}}(z_L)&=&\int\!\dr\chi~\frac{\dr^2V}{\dr\chi
 d\Omega}\int\!\dr M\frac{\dr n}{\dr M}S(M, \chi;z_L),
% =
% \int\!\dr\chi~\chi^2\int\!\dr M\frac{\dr n}{\dr M}S(M,\chi; z_L),
%\nonumber\\
% &\simeq & \chi_L^2\Delta\chi\int\!\dr M\frac{\dr n}{\dr M}S(M,z_L)\nonumber\\
% &=&\chi_L^2 \Delta \chi \bnh(z_L)
\end{eqnarray}
%
where $\dr^2V/\dr\chi d\Omega=\chi^2$ for a flat-geometry universe,
$\chi(z)$ is the comoving angular distance to redshift $z$, $dn/\dr M$ is
the three-dimensional number density of halos with masses $[M,\dr M]$ at
$z(=z(\chi))$, and $S(M,\chi;z_L)$ is the selection function of halos at
a bin of redshift around $z_L$. Note that the radial and
angular-diameter distances are equivalent for a flat-geometry universe.
For simplicilty, assuming that mass and redshift for each halo is
available, we consider the following, trivial selection function,
defined as
%
\begin{equation}
 S(\chi,M;z_L)=\Theta(\chi-\chi_L+\Delta\chi/2)\Theta(\chi_L+\Delta\chi/2-\chi)
\tilde{S}(M),
%\tilde{S}(M), 
\end{equation}
%
where $\Theta(x)$ is the Heviside-step function, defined as
$\Theta(x)=1$ if $x>0$ otherwise $\Theta(x)=0$, $\chi_L$ is the radial
comoving distance to $z_L$, $\chi_L\equiv \chi(z_L)$, and
$\Delta\chi$ is the radial bin width. $S(\chi,M;z_L)$ is nonzero in the
radial range of $\chi_L-\Delta\chi/2\le \chi\le \chi_L+\Delta\chi/2$.
$\tilde{S}(M)$ is the selection of halo masses. We don't specify the
form of $\tilde{S}(M)$ to keep generality of our discussion.

If a finite-volume survey region is embedded into a coherent density
contrast, which we hereafter call the background mode $\delta_\br$, the
cumulative number of halos is modulated as
%
\begin{equation}
 \hat{N}_\bh(z_L;\delta_\br)=\int\!\dr\chi~\chi^2\int\!\dr M~\frac{\dr n}{\dr M}S(M,\chi;z_L)
  \left[1+b(M)\delta_\br\right]=\bar{N}_{\bh}(z_L)(1+\bar{b}_1\delta_\br)
\end{equation}
%
where $b(M)$ is the linear halo bias and $\bar{b}_1$ is the average
bias over the halo mass function:
%
\begin{equation}
 \bar{b}_1\equiv
  \frac{1}{\bar{N}_\bh(z_L)}\int\!\dr\chi~\chi^2\int\!\dr M\frac{\dr n}{\dr M}S(M,\chi;z_L)b(M)
  =\frac{1}{\bar{N}_\bh(z_L)}\int_{\chi_L}\!\dr\chi~\chi^2{n}_\bh^0(\chi),
\end{equation}
%
where the integration $\int_{\chi_L}\!\dr\chi$ denotes the radial
integration over the range $[\chi_L-\Delta\chi/2,\chi_L+\Delta\chi/2]$,
and we introduced the collapsed notation:
%
\begin{equation}
 {n}_{\bh}^\beta(\chi)\equiv \int\!\dr M~\frac{\dr n}{\dr M}\tilde{S}(M)\left[b(M)\right]^\beta.
  \label{eq:def_S}
\end{equation}
%
The background density mode is defined in terms of the survey window
funciton $W(\chi,\btheta)$ as
%
\begin{equation}
\delta_\br\equiv
 \frac{1}{V_S}\int\!\dr\chi~\chi^2\int_{\Omega_s}\!\dr^2\btheta~ W(\chi,\btheta)
 \deltam(\chi,\chi\btheta).
\end{equation}
%
Note that the survey window generally includes both the radial and
angular windows.  If the survey window has a simple geometry, without
any masked region, the survey volume for the redshift slice around
$z_L$, $V_S$, is simply given as $V_S\simeq \chi^2_L\Delta\chi\Omega_S$, where
$\Omega_S$ is the survey area.

The projected number density fluctuation field of halos in the radial
bin around $z_L$ is given as
%
\begin{eqnarray}
 \delta^{\rm
  2D}_{\mathrm{h}}(\bx_\perp,z_L)&=&\frac{1}{\hat{N}_{\mathrm{h}}}\int_{\chi_L}\!\dr\chi~
  %  \frac{\dr^2V}{\dr\chid\Omega}
  \chi^2
  \!\int\!\dr M\frac{\dr n}{\dr M}\tilde{S}(M)\deltah(\chi,\bx_\perp;
  M),
%  \nonumber\\
% &\simeq &
  % \frac{1}{\bnh(z_L)}\int\!\dr M\frac{\dr n}{\dr M}S(M;z_L)\deltah(\chi,\bx_\perp; M).
\end{eqnarray}
%
where $\bx_\perp$ is the projected position vector of each halo in the
perpendicular plane to the line-of-sight.
%Hence the Fourier mode of the
%halo distribution, needed for the shear-halo cross-correlation, is given as
%%
%\begin{eqnarray}
% \hat{\delta}^{\rm 2D}_{\bh}(\bk_\perp;z_L)&=&\int\!\dr^2\bx_\perp
%  \delta^{\rm 2D}_{\bh}(\bx_\perp;
%  z_L)e^{-i\bk_\perp\cdot\bx_\perp}\nonumber\\
% &=& \frac{1}{\hat{N}_{\mathrm{h}}}\int\!\dr\chi~\frac{\dr^2V}{\dr\chi
%  d\Omega}\!\int\!\dr M\frac{\dr n}{\dr M}S(M;z_L)\int\!\frac{\dr k_\parallel}{2\pi}
%  \deltah(\bk_\perp,k_\parallel;M, z_L)e^{ik_\parallel\chi}.
%\end{eqnarray}

Using the Limber's appxoimation \citep{Limber:54}, we can express the
projected power spectrum of halos as
%
\begin{eqnarray}
 \phh^{\rm 2D}(k_{\perp};z_L)&=&\frac{1}{\hat{N}_{\bh}(z_L)^2}
  \int_{\chi_L}\!\dr\chi~\chi^4
  \int\!\dr M~\frac{\dr n}{\dr M}\tilde{S}(M)
  \int\!\dr M'~\frac{\dr n}{\dr M'}\tilde{S}(M')
  \phh(k_\perp;M,M',z_L)\nonumber\\
 &=&\frac{1}{\hat{N}_{\bh}(z_L)^2}
  \int_{\chi_L}\!\dr\chi~\chi^4\left[n^1_{\bh}(\chi)\right]^2\pml(k_\perp;\chi)
\end{eqnarray}
%
where $\phh(k;M,M',\chi)$ is the three-dimensinal power spectrum between
two halo of masses $M$ and $M'$ at redshift $z=z(\chi)$. In the second
line on the r.h.s., we assumed that the halo power spectrum is given as
$\phh(k,M,M')\simeq
b(M)b(M')P^{\rm lin}_{\rm m}(k)$ in the halo model approach, where
$\pml(k)$ is the linear mass power spectrum. In the following we will
often denote $k$ instead of $k_\perp$ for notational simplification.
%Note that the wavenumber
%$\k$ in the above equation is from wavevectors $\bk_\perp$ perpendicular
%to the line-of-sight direction due to the Limber's approximation.
If we
further assume a thin redshift slice, $\Delta\chi/\chi_L\ll 1$, the halo
power spectrum can be further simplified as
%
\begin{eqnarray}
 \phh(k_\perp;z_L)\simeq \frac{\bar{b}_1^2}{\Delta \chi}P^{\rm lin}_{\rm m}(k).
\end{eqnarray}
%
%where
%%
%\begin{equation}
%\bar{b}_1\equiv \frac{1}{\bnh(z_L)}\int\!\!\dr M~\frac{\dr n}{\dr M}S(M,z_L)b(M).
%\end{equation}



From the above equation, we can find the power spectrum of stacked
lensing as
%
\begin{equation}
 P_{\dsigma}(k)=\frac{\bar{\rho}_{\rm m0}}{\hat{N}_\bh(z_L)}
  \int_{\chi_L}\!\dr\chi~\chi^2
  \int\!\dr M~\frac{\dr n}{\dr M}\tilde{S}(M)P_{\rm hm}(k;M).
\end{equation}
%
The average excess surface mass density profile, $\ave{\dsigma}(R)$,
is a more direct observable of the stacked lensing, and is given as
%
\begin{equation}
 \ave{\dsigma}(R)=\int\!\frac{k\dr k}{2\pi}P_{\dsigma}(k)J_2(kR).
\end{equation}
%
If we use the halo model, we can express the power spectrum by a sum of
the 1- and 2-halo terms:
%
\begin{equation}
 P_{\dsigma}(k)=P^{\rm 1h}_{\dsigma}(k)+P^{\rm 2h}_{\dsigma}(k),
\end{equation}
%
with
%
\begin{eqnarray}
 P^{\rm
  1h}_{\dsigma}(k)&\equiv &
  \frac{\bar{\rho}_{\rm m0}}{\hat{N}_\bh(z_L)}\int_{\chi_L}\!\dr\chi ~\chi^2
  \int\!\dr M~\frac{\dr n}{\dr M}S(M,z_L)M|u_M(k;z_L)|
  =\frac{\bar{\rho}_{\rm m0}}{\hat{N}_\bh(z_L)}\int_{\chi_L}\!\dr\chi ~\chi^2I^0_1(k;\chi),\nonumber \\
 P^{\rm 2h}_{\dsigma}(k)&=&
 \frac{\bar{\rho}_{\rm m0}}{\hat{N}_\bh(z_L)}\int_{\chi_L}\!\dr\chi ~\chi^2
  \int\!\dr M~\frac{\dr n}{\dr M}\tilde{S}(M)
  P_{\rm hm}^{2h}(k;M,z_L)
=\frac{\bar{\rho}_{\rm m0}}{\hat{N}_\bh(z_L)}\int_{\chi_L}\!d\chi
~\chi^2 n_\bh^1(\chi)\pml(k;\chi)
\simeq \bar{b}_1\bar{\rho}_{\rm m0}P^{\rm lin}_{\rm m}(k),\nonumber \\
\end{eqnarray}
%
where we introduced the notation:
%
\begin{equation}
 I^\beta_\mu(k_1,\cdots,k_\mu)\equiv \int\!\dr M\frac{\dr n}{\dr M}\tilde{S}(M)
  \beta_1^\beta\prod_{i=1}^\mu u_M(k_i),
\end{equation}
%
where $u_M(k)$ is the Fourier transform of the average mass density
profile for halos of mass $M$.

\end{document}