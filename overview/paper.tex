%%%%%%%%%%%%%%%%%%%%%%%%%%%%%%%%%%%%%%%%%%%%%%%%%%%%%%%%%%%%%%%%%%%%%%%%%%
% HEADER
%%%%%%%%%%%%%%%%%%%%%%%%%%%%%%%%%%%%%%%%%%%%%%%%%%%%%%%%%%%%%%%%%%%%%%%%%%

\pdfoutput=1

\documentclass[iop, apjl, twocolappendix, numberedappendix]{emulateapj}

\usepackage{xspace}
\usepackage{amsmath}
\usepackage{framed} 
\usepackage{txfonts}
\usepackage{epstopdf}
\usepackage{color}
\usepackage{rotating}
\usepackage{natbib}
\usepackage{ulem}
\usepackage{xspace}
\usepackage[colorlinks=true,urlcolor=blue,linkcolor=blue,citecolor=blue]{hyperref}



\special{papersize=8.5in,11in}
\setlength{\pdfpageheight}{\paperheight}
\setlength{\pdfpagewidth}{\paperwidth}
\newdimen\hssize
\hssize=8.4truecm
\newdimen\hdsize
\hdsize=16.8truecm

\input{commands}
% Frequently used expressions

\def\rs{r_{\rm s}}
\def\rhos{\rho_{\rm s}}
\def\se{s_{\rm e}}
\def\ftrans{f_{\rm trans}}
\def\mstar{M_{\ast}}
\def\mhalo{M_{\rm halo}}

\def\rdelta{R_{\Delta}}
\def\mdelta{M_{\Delta}}

\def\rsp{R_{\rm sp}}
\def\msp{M_{\rm sp}}

\def\mfrs{M_{<4r_\rms}}

\def\rinfall{R_{\rm infall}}
\def\minfall{M_{\rm infall}}

\def\fpe{f_{\rm bdry}}
\def\fbar{f_{\rm bar}}

\def\rhoc{\rho_{\rm c}}
\def\rhom{\rho_{\rm m}}
\def\rhoref{\rho_{\rm ref}}

\def\cvir{c_{\rm vir}}
\def\cgal{c_{\rm gal}}

\def\omm{\Omega_{\rm m}}
\def\pcen{p_{\rm cen}}
\def\pmem{p_{\rm mem}}
\def\sigmag{\Sigma_{\rm g}}
\def\rhog{\rho_{\rm g}}
\def\rhoginner{\rho_{\rm g}^{\rm inner}}
\def\rhogouter{\rho_{\rm g}^{\rm outer}}
\def\rt{r_{\rm t}}
\def\rout{r_{\rm out}}
\def\mdm{m_{\rm dm}}

\def\nsat{N_{\rm sat}}
\def\nsattod{N_{\rm sat}^{\rm 2d}}
\def\nsatthd{N_{\rm sat}^{\rm 3d}}
\def\rmem{R_{\rm mem}}
\def\rmemtod{R_{\rm mem}^{\rm 2d}}
\def\rsptod{R_{\rm sp}^{\rm 2d}}
\def\rspthd{R_{\rm sp}^{\rm 3d}}
\def\mtom{M_{\rm 200m}}
\def\mtomthd{M_{\rm 200m}^{\rm 3d}}
\def\rtom{R_{\rm 200m}}
\def\ctom{c_{\rm 200m}}

\def\mvir{M_{\rm vir}}
\def\mtoc{M_{\rm 200c}}
\def\rtoc{R_{\rm 200c}}
\def\ctoc{c_{\rm 200c}}

\def\mfoc{M_{\rm 500c}}
\def\rfoc{R_{\rm 500c}}
\def\cfoc{c_{\rm 500c}}

\def\mtfc{M_{\rm 2500c}}
\def\rtfc{R_{\rm 2500c}}
\def\ctfc{c_{\rm 2500c}}

\def\vpeak{V_{\rm peak}}
\def\cmsqpg{{~\rm cm^2g^{-1}}}


\newcommand{\redm}{redMaPPer}
\newcommand{\redms}{redMaPPer }

\def\ave#1{\left\langle #1 \right\rangle}

\input{citation_fix}


\def\avrg#1{\left\langle #1 \right\rangle}
\def\avrgb#1{\left\langle #1 \right\rangle_\textrm{b}}

\newcommand{\mtrv}[1]{{\textcolor{red}{#1}}}

\shorttitle{Halo Emulator for Halo Statistics}
\shortauthors{Nishimichi et al.}

%\journalinfo{The Astrophysical Journal, {\rm 789:1 (18pp), 2014 July 1}}
%\submitted{Received 2014 January 6; accepted 2014 April 14; published 2014 June 9}
\slugcomment{To be submitted to the Astrophysical Journal}

\begin{document}

%%%%%%%%%%%%%%%%%%%%%%%%%%%%%%%%%%%%%%%%%%%%%%%%%%%%%%%%%%%%%%%%%%%%%%%%%%
% EPS OR PDF FIGURES
%%%%%%%%%%%%%%%%%%%%%%%%%%%%%%%%%%%%%%%%%%%%%%%%%%%%%%%%%%%%%%%%%%%%%%%%%%

\def\figdir{.}
\def\figext{pdf}

%%%%%%%%%%%%%%%%%%%%%%%%%%%%%%%%%%%%%%%%%%%%%%%%%%%%%%%%%%%%%%%%%%%%%%%%%%
% TITLE ETC
%%%%%%%%%%%%%%%%%%%%%%%%%%%%%%%%%%%%%%%%%%%%%%%%%%%%%%%%%%%%%%%%%%%%%%%%%%


\title{Halo emulator: fast computation of halo statistics}
\author{
Takahiro Nishimichi \altaffilmark{1}, 
....
}

\affil{
$^1$ Kavli Institute for the Physics and Mathematics of the Universe (WPI),
Tokyo Institutes for Advanced Study, The University of Tokyo,\\ 5-1-5
Kashiwanoha, Kashiwa-shi, Chiba, 277-8583, Japan;{\tt takahiro.nishimichi@ipmu.jp}}



%%%%%%%%%%%%%%%%%%%%%%%%%%%%%%%%%%%%%%%%%%%%%%%%%%%%%%%%%%%%%%%%%%%%%%%%%%
% ABSTRACT
%%%%%%%%%%%%%%%%%%%%%%%%%%%%%%%%%%%%%%%%%%%%%%%%%%%%%%%%%%%%%%%%%%%%%%%%%%

 \begin{abstract}
  We develop a emulator of halo statistics (halo mass function, halo
  auto-spectrum, and halo-matter cross correlation) that are given as a
  function of halo mass, redshift and cosmological model, carefully
  calibrated using a suite of $N$-body siulations assuming cold dark
  matter dominated structure formation scenario. 
 \end{abstract}

\keywords{}

%%%%%%%%%%%%%%%%%%%%%%%%%%%%%%%%%%%%%%%%%%%%%%%%%%%%%%%%%%%%%%%%%%%%%%%%%%
% INTRODUCTION
%%%%%%%%%%%%%%%%%%%%%%%%%%%%%%%%%%%%%%%%%%%%%%%%%%%%%%%%%%%%%%%%%%%%%%%%%%

\section{Introduction}
\label{sec:intro}

\section{Results}

\mtrv{The results should include halo mass function, halo auto-spectra,
 halo-matter cross-spectra, matter power spectrum, bias function $b(k)$,
 cross-correlation function $r(k)$, ... as a function of halo mass and
 redshift, and cosmological model.}


\section{Measurement of Cluster-galaxy lensing}

\bibliographystyle{apj}

\end{document}
